\startsection[title=Conclusions]

    An optimal estimation method for the retrieval of vertical profiles of
    temperature and atmospheric water content from brightness temperature
    measurements of a microwave radiometer has been implemented. Part of the
    method is a radiative transfer model for the microwave region which was
    also developed based on existing models for absorption of microwave
    radiation by oxygen, water vapor and cloud liquid water. The radiative
    transfer model is written in a high-level programming language, uses
    automatic differentiation for the calculation of exact linearization of the
    radiative transfer and has been found to be sufficiently accurate for use in
    the optimal estimation scheme.

    Retrieval results from this scheme suggest an average accuracy of
    temperature retrievals better than 1 K throughout the troposphere. The
    retrieved temperature profiles benefit greatly from forecasts of a
    regional weather prediction model in the middle and upper troposphere
    while the information contained in the radiometer observations provide
    complementary information in the lower atmosphere where the numerical model
    has its largest uncertainties.
    
    Two case studies of nocturnal temperature inversions showed that the
    optimal estimation scheme is able to accurately resolve such structures
    if the prior information is adequate. It was found that the atmospheric
    features exhibited by the prior are very important for the retrieval
    of an elevated temperature inversion likely because it is difficult for
    the optimal estimation scheme to introduce a fine scale feature into
    the first guess due to the relatively smooth weighting functions of the
    radiometer channels. A general deviation of the prior from the true
    atmospheric state in terms of absolute values is easier to correct
    for the technique.

    No major advancements in the retrieval of atmospheric water content were
    achieved with the considered optimal estimation scheme.
    This can partially be attributed
    to the smaller number of radiometer observations available in the
    humidity-sensitive region of the microwave spectrum and to larger
    uncertainties in the radiative transfer due to a strong dependency on
    the temperature retrieval and the effects of cloud liquid water absorption.

\stopsection


\startsection[title=Outlook]

    Because a lot of time was spent on the  development of the radiative
    transfer model and the setup of the optimal estimation procedure, results
    presented here were rather general and only basic variants of the methods
    were used. Particularly the application to observations from HATPRO
    suffered under the small number of cases included in the performance
    assessment.

    The foundation for experiments with the optimal estimation scheme has been
    laid and the case studies have shown that the method has much promise for 
    future applications. The inclusion of additional information into the
    retrieval has been the focus of previous studies in Innsbruck and should
    be continued also with the optimal estimation technique whose prior
    distribution is the ideal place for such information.

    A region between 1.5 and 3 km height above ground level was found in the
    performance evaluation where temperature retrieval accuracy is worse than
    in other regions of the troposphere. In this region COSMO-7 forecasts are
    not highly accurate yet and the information content from radiometer
    observations has already decreased substantially. These heights are however
    in the reach of sensors placed on mountains and the observations that
    \cite[Massaro2013,Meyer2016] included in their regression retrievals could
    help to improve the accuracy of this region specifically. Forecasts from
    a NWP model with higher horizontal and vertical resolution can also
    help to improve the accuracy in this region and could also provide priors
    that better resolve fine-scale features of the atmospheric state.

    Aside from the method itself there is also a demand for better evaluation
    tools with which the accuracy of retrieval schemes is measured. Standard
    deviations and biases have little significance for the performance in
    specific atmospheric situations. It would be nice to have feature-based
    measures to better assess the resolution of inversions by the retrieved
    profiles. Standard deviations are also not universally comparable
    due to differences in the natural variability of atmospheric state
    variables between locations \cite[authoryears][Tan2011].

    If the implemented radiative transfer model should be a part of future
    research in Innsbruck is questionable. While the model works, future
    studies could benefit from better comparability to other research if an
    established model like ARTS \cite[authoryears][Buehler2005] or the
    hopefully soon available RTTOV-gb \cite[authoryears][DeAngelis2016] is
    used.  These models are also closer to the current state of research in
    atmospheric radiative transfer because experts in this field contribute to
    the development.

\stopsection

