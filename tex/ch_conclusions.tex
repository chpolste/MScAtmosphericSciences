\startsection[title=Conclusions]

\stopsection


\startsection[title=Future Work]

    How to improve results.

    Bias correction, better priors, model development, additional profile
    information (e.g. TODO: Height attribution of clouds in the valley is
    possible without ceilometers or cloud radar due to terrain markers.), idea
    and challenges of continuous retrieval

    Radiosondes debated as being "truth" due to problems with measurement
    errors, drift (important for higher altitudes \cite[Crewell2007] and short
    term temporal variations \cite[Gueldner2001,Guledner2013,Tan2011].

    RMS as measure of error not generally comparable. Hong Kong has high
    humidity all year so rms is naturally higher \cite[Tan2011].
    Also size and composition of training and test data sets as well as priors
    vary strongly between studies making it difficult to compare individual
    methods in isolation from the data that they are driven by.

    It is probably easier to use ARTS or the hopefully soon to be released
    RTTOV-gb instead of fixing all issues with the MWRTM prototype. This will
    also increase confidence in the results and benefit comparability to other
    studies using the same models. The inability to reproduce model comparison
    results from other authors shows that there is more to radiative transfer
    than the absorption model. This aspect is talked about only little in
    the literature.

\stopsection

