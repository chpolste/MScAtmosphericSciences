The location of interest here is Innsbruck, located in the Inn-valley of the
Alps at approximately 600 m, surrounded by mountains of up to 2300 m height. It
provides an challenging environment for radiometer retrievals as the boundary
layer evolution is strongly affected by the topography and most NWP models
resolve the terrain insufficiently.

\startsection[title=HATPRO]

    The Institute of Atmospheric and Cryospheric Sciences at the University of
    Innsbruck operates a passive microwave radiometer on the roof of one of the
    university's buildings as part of the Innsbruck Box project. The instrument
    is a Humidity and Temperature Profiler (HATPRO), built by the Radiometer
    Physics GmbH \cite[authoryears][Rose2005]. It obtains brightness
    temperatures at 14 channels with a resolution of 1 s. The receiver antenna
    can be rotated to allow elevation scanning. Additionally, HATPRO measures
    environmental pressure, temperature and humidity.
    
    Frequencies in water vapor band (K-band): 22.24, 23.04, 23.84, 25.44,
    26.24, 27.84, 31.40 GHz.
    Frequencies in oxygen band (V-band): 51.26, 52.28, 53.86, 54.94, 56.66,
    57.30, 58.00 GHz.

    Boundary layer scans every ??? minutes, otherwise zenith measurements.
    Angles of boundary layer scans (deviation from zenith): 0°, 60°, 70.8°,
    75.6°, 78.6°, 81.6°, 83.4°.

    Raw data available from 07-2015 to 10-2015 (???), this period provides
    little data gaps and maximum overlap with other data sources.

\stopsection

\placefigure[bottom][fig:map_ibk]
        {Data locations in the Inn-valley around Innsbruck. 1: Innsbruck
        Airport (577 m), radiosonde launch location. 2: HATPRO location with
        elevation scanning direction (612 m). 3: COSMO7 model grid point.
        4 - 7: Nordkette slope stations Alpenzoo (710 m), Hungerburg (920 m),
        Rastlboden (1220 m) and Hafelekar (2270 m). Map source:
        http://www.basemap.at}
        {\externalfigure[map][width=\textwidth]}


\startsection[title=Radiosonde Climatology]

    Treated as reference data set, radiosondes have high accuracy but drift
    and time-lag problem. 

    Distance airport to university is approximately 2.5 km.

    What available, which resolution. Applied ice-liquid-split of water outside
    of model. KARSTENS ET AL liquid water content.

\stopsection

\startsection[title=COSMO7 Simulated Radiosoundings]

    Determine prior covariance matrix by comparison with raso.

    In clouds always 100 \% RH. Partition specific cloud content into liquid
    and solid parts using

    \startformula
        \QLIQ = \QCLOUD \startcases
            \NC 0 \MC 233.15 \KELVIN \le \TEMP \NR
            \NC \frac{\TEMP - 233.15 \KELVIN}{40 \KELVIN}
                \MC 233.15 \KELVIN \lt \TEMP \lt 273.15 \KELVIN \NR
            \NC 1 \MC 273.15 \KELVIN \le \TEMP \NR
        \stopcases
    \stopformula

    Ice content is thrown away.

    Problem: model topography end above 1 km (???) but Innsbruck is at
    approximately 600 m, therefore the profiles have to be extrapolated to the
    surface. Done by connecting lowest COSMO7 level temperature with radiometer
    surface measurement, then recalculating the pressure using the hydrostatic
    equation.

\stopsection


\startsection[title={Nordkette Slope Measurements}]

    Data by ZAMG, previously used by Daniel Meyer, describe location of
    stations and their height. Hint at use as likelihoof function/prior.

\stopsection

