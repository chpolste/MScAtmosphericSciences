The discussed optimal estimation and linear regression methods are applied
to retrievals of temperature and humidity profiles based on data collected
in Innsbruck. A radiosonde climatology provides reference profiles for
the evaluation of synthetic performance of the retrieval techniques based
on simulated brightness temperatures and for an application to actual
radiometer data. Additional prior information is obtained from vertical
profiles forcasted by a numerical weather prediction model. Figure
\in[fig:map_ibk] gives an overview of the Innsbruck region and the locations of
all data sources.


\startsection[title={Review of the Retrieval Implementation}]

    

\stopsection


\placefigure[top][fig:map_ibk]
        {Data locations in the Inn-valley around Innsbruck. 1: Innsbruck
        Airport (577 m), radiosonde launch location. 2: HATPRO location with
        elevation scanning direction (612 m). 3: COSMO7 model grid point.
        4 - 7: Nordkette slope stations Alpenzoo (710 m), Hungerburg (920 m),
        Rastlboden (1220 m) and Hafelekar (2270 m). Map source:
        http://www.basemap.at}
        {\externalfigure[map][width=\textwidth]}


\startsection[title={Radiosonde Climatology},reference=ch:rasoclim]

    Radiosondes are launched regularly once per day from Innsbruck airport
    at the location of marker 1 in Figure \in[fig:map_ibk]. Launch time is
    usually between 01:00 and 04:00 UTC, which corresponds to 02:00 to 05:00
    local time in winter and 03:00 to 06:00 local time during daylight saving
    time. Here, the climatological dataset of temperature, humidity and
    pressure spanning the years 1999 to 2005 and 2009 to 2012, previously used
    by \cite[Massaro2015] and \cite[Meyer2016], is extended with profiles
    obtained between 2013 and 2016.  In total, data from 3561 radiosonde are
    available which are separated into a training data set of 3296 profiles and
    a test data set of 265 profiles which contains all data between February
    2015 and January 2016. This partitioning results in the largest overlap of
    test data with other data sources.

    Because liquid water content is not measured by any instrument during
    a balloon sounding, cloud water content is calculated by the empirical
    cloud model of \cite[Karstens1994] based on humidity and temperature
    information. This model is frequently used for microwave radiometer
    retrieval studies, e.g. by \cite[Lohnert2012,Martinet2015].

    The profiles of pressure, temperature, humidity and liquid water content
    are linearly interpolated to the 50 levels of the retrieval grid. Figure
    \in[fig:raso_prior] shows the climatology's mean profile of temperature and
    total water content together with intervals of ±1 and ±2 standard
    deviation. Temperature variability is almost constant throughout the
    troposphere, while water variability is greater near the surface and
    vanishes at the mean height of the tropopause. The total water content
    uncertainty intervals are asymmetric because the underlying distribution is
    a Gaussian in logarithmic space (see section \in[ch:statevector]).

    Whenever the accuracy of a retrieval method is evaluated in subsequent
    sections, these radiosonde data are the reference that retrieved profiles
    are compared to. It must be noted that radiosondes provide point
    measurements, are affected by horizontal drift and take a substantial amout
    of time to reach their final height. These properties of balloon-bourne
    sensors raise concerns regarding the representativity of the measured
    vertical profiles for a fixed location. Also, only 4 \% of all radiosonde
    profiles were obtained during day time which has to be condsidered when
    assessing the applicability of retrieval methods based on the climatology.
    Inaccuracies due to these issues are however tolerated due to the lack of
    a better alternative.

    \placefigure[top][fig:raso_prior]
            {The radiosonde climatology of temperature (left) and total water
            content (vapor and liquid water) derived from radiosondings from
            Innsbruck airport. The black lines indicate the climatological
            mean value, grey shadings mark the one (dark) and two (light)
            standard deviation interals obtained from sampling the
            distribution.
            }
            {\externalfigure[raso_prior][width=\textwidth]}

\stopsection


\startsection[title=COSMO7 Simulated Radiosoundings]

    One year of simulated profiles (from ??? to ???) for gridpoint near
    Innsbruck available. Lead times in 6 h steps and up to +30 h are used. To
    match radiosonde ascent times and other retrieval times, profiles are
    linearly interpolated between the nearest valid times of model forecasts.

    Problem: model topography end above 1 km (???) but Innsbruck is at
    approximately 600 m, therefore the profiles have to be extrapolated to the
    surface. Easy solution considered was connecting COSMO7 profile to
    radiometer surface measurement but model contains boundary layer processes
    which would then be elevated. Instead profiles were stretched downward by
    replacing the height grid such that COSMO surface is IBK surface. Because
    it is assumed that the forecasts become less displaced farther away from
    the surface, lower levels are stretched more than upper levels. Because
    the stretching does not take temperature gradients into account and there
    might be a possible model bias, the mean bias wrt radiosondes is removed
    from the dataset. This can be understood as a simple MOS correction.
    Important implication: profiles are not independent of raso test data set
    any more. Better would be to use model uncertainty determined by.
    assimilation system (REFERENCE) but that was not available.

    Determine prior covariance matrix by comparison with raso: Figure
    \in[fig:cosmo7_prior]. Uncertainty large in lower troposphere, but smaller
    outside the boundary layer as expected (confirms assumptions made during
    stretching). Example profile shows that model resolution is fine enough
    that inversions may be represented in the model.

    In clouds always 100 \% RH. Partition specific cloud content into liquid
    and solid parts using

    \startformula
        \QLIQ = \QCLOUD \startcases
            \NC 0 \MC 233.15 \KELVIN \le \TEMP \NR
            \NC \frac{\TEMP - 233.15 \KELVIN}{40 \KELVIN}
                \MC 233.15 \KELVIN \lt \TEMP \lt 273.15 \KELVIN \NR
            \NC 1 \MC 273.15 \KELVIN \le \TEMP \NR
        \stopcases
    \stopformula

    Ice content is thrown away. (REFRENCE)

    \placefigure[top][fig:cosmo7_prior]
            {A COSMO7 prior distribution of temperature (left) and specific
            humidity (right) for the lower half of the troposphere, valid at
            2015-02-06 03:02 UTC. The means (black lines) are interpolated from
            lead times +00 and +06 of the 2015-02-06 00:00 UTC run. The grey
            shadings show the one (dark) and two (light) standard deviation
            intervals obtained by sampling.
            }
            {\externalfigure[cosmo7_prior][width=\textwidth]}

\stopsection


\startsection[title=HATPRO,reference=ch:hatpro]

    The Institute of Atmospheric and Cryospheric Sciences at the University of
    Innsbruck operates a passive microwave radiometer on the roof of one of the
    university's buildings as part of the Innsbruck Box project. The instrument
    is a Humidity and Temperature Profiler (HATPRO), built by the Radiometer
    Physics GmbH \cite[authoryears][Rose2005]. It obtains brightness
    temperatures at 14 channels with a resolution of 1 s. The receiver antenna
    can be rotated to allow elevation scanning. Additionally, HATPRO measures
    environmental pressure, temperature and humidity.

    Distance airport to university is approximately 2.5 km.
    Radiosond launches 577 m.

    View downvalley in opposite direction of airport.
    
    Frequencies in water vapor band (K-band): 22.24, 23.04, 23.84, 25.44,
    26.24, 27.84, 31.40 GHz.
    Frequencies in oxygen band (V-band): 51.26, 52.28, 53.86, 54.94, 56.66,
    57.30, 58.00 GHz.

    TODO: channel selection for optimal estimation: fulfilment of Gaussian
    assumption \cite[Martinet2015].

    Boundary layer scans every 10 minutes, otherwise zenith measurements.
    Angles of boundary layer scans (deviation from zenith): 0°, 60°, 70.8°,
    75.6°, 78.6°, 81.6°, 83.4°.

    Raw data available from 07-2015 to 10-2015 (???), this period provides
    little data gaps and maximum overlap with other data sources. Averaging in
    ?? min intervals of zenith measurements. Bias (wrt to numerical model)
    correction by comparison with radiosonde climatology. This means that
    measurements are not independent of referenc data anymore but else too
    little data are available. In operational applications one might do rolling
    updates to the bias correction as well. Profits non-verifiable data during
    the day.

\stopsection

