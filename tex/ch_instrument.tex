Introduce map with data locations, mention every instrument.


\startsection[title={Applied Retrieval techniques}]

    Which are used, short information about choice of parameters, algorithms,
    components, etc.

    All profiles cut off at 15 km. 50 level retrieval grid starting at 612 m,
    log spacing.

\stopsection


\startsection[title=HATPRO,reference=ch:hatpro]

    TODO: avoid repetition with introduction.
    The Institute of Atmospheric and Cryospheric Sciences at the University of
    Innsbruck operates a passive microwave radiometer on the roof of one of the
    university's buildings as part of the Innsbruck Box project. The instrument
    is a Humidity and Temperature Profiler (HATPRO), built by the Radiometer
    Physics GmbH \cite[authoryears][Rose2005]. It obtains brightness
    temperatures at 14 channels with a resolution of 1 s. The receiver antenna
    can be rotated to allow elevation scanning. Additionally, HATPRO measures
    environmental pressure, temperature and humidity.
    
    Frequencies in water vapor band (K-band): 22.24, 23.04, 23.84, 25.44,
    26.24, 27.84, 31.40 GHz.
    Frequencies in oxygen band (V-band): 51.26, 52.28, 53.86, 54.94, 56.66,
    57.30, 58.00 GHz.

    Boundary layer scans every ??? minutes, otherwise zenith measurements.
    Angles of boundary layer scans (deviation from zenith): 0°, 60°, 70.8°,
    75.6°, 78.6°, 81.6°, 83.4°.

    Raw data available from 07-2015 to 10-2015 (???), this period provides
    little data gaps and maximum overlap with other data sources. Averaging in
    ?? min intervals of zenith measurements. Bias (wrt to numerical model)
    correction by comparison with radiosonde climatology. This means that
    measurements are not independent of referenc data anymore but else too
    little data are available. In operational applications one might do rolling
    updates to the bias correction as well. Profits non-verifiable data during
    the day.

\stopsection

\placefigure[top][fig:map_ibk]
        {Data locations in the Inn-valley around Innsbruck. 1: Innsbruck
        Airport (577 m), radiosonde launch location. 2: HATPRO location with
        elevation scanning direction (612 m). 3: COSMO7 model grid point.
        4 - 7: Nordkette slope stations Alpenzoo (710 m), Hungerburg (920 m),
        Rastlboden (1220 m) and Hafelekar (2270 m). Map source:
        http://www.basemap.at}
        {\externalfigure[map][width=\textwidth]}


\startsection[title=Radiosonde Climatology]

    Treated as reference data set, radiosondes have high accuracy but drift
    and time-lag problem. 

    Distance airport to university is approximately 2.5 km.

    What available, which resolution, which sensor? Interpolation to retrieval
    grid. How many night, how many day sondes? KARSTENS ET AL liquid water
    content.

    Reference Figure \in[fig:raso_prior]. Training dataset: 3229 profiles
    between ?? and ??. Test dataset 249/250? sondes between ?? and ??. Choice
    of test and training datasets is data-driven, this way maximum
    availablility overlap is achieved between raso, cosmo, nordkette and hatpro
    measurements are available as well.


    \placefigure[top][fig:raso_prior]
            {The radiosonde climatology of temperature (left) and specific
            humidity (including LWC) used as a prior in subsequent retrievals.
            The black lines indicate the climatological mean, the grey shadings
            the one (dark) and two (light) standard deviation interals obtained
            by sampling the distribution.
            }
            {\externalfigure[raso_prior][width=\textwidth]}

\stopsection


\startsection[title=COSMO7 Simulated Radiosoundings]

    One year of simulated profiles (from ??? to ???) for gridpoint near
    Innsbruck available. Lead times in 6 h steps and up to +30 h are used. To
    match radiosonde ascent times and other retrieval times, profiles are
    linearly interpolated between the nearest valid times of model forecasts.

    Problem: model topography end above 1 km (???) but Innsbruck is at
    approximately 600 m, therefore the profiles have to be extrapolated to the
    surface. Easy solution considered was connecting COSMO7 profile to
    radiometer surface measurement but model contains boundary layer processes
    which would then be elevated. Instead profiles were stretched downward by
    replacing the height grid such that COSMO surface is IBK surface. Because
    it is assumed that the forecasts become less displaced farther away from
    the surface, lower levels are stretched more than upper levels. Because
    the stretching does not take temperature gradients into account and there
    might be a possible model bias, the mean bias wrt radiosondes is removed
    from the dataset. This can be understood as a simple MOS correction.
    Important implication: profiles are not independent of raso test data set
    any more. Better would be to use model uncertainty determined by.
    assimilation system (REFERENCE) but that was not available.

    Determine prior covariance matrix by comparison with raso: Figure
    \in[fig:cosmo7_prior]. Uncertainty large in lower troposphere, but smaller
    outside the boundary layer as expected (confirms assumptions made during
    stretching). Example profile shows that model resolution is fine enough
    that inversions may be represented in the model.

    In clouds always 100 \% RH. Partition specific cloud content into liquid
    and solid parts using

    \startformula
        \QLIQ = \QCLOUD \startcases
            \NC 0 \MC 233.15 \KELVIN \le \TEMP \NR
            \NC \frac{\TEMP - 233.15 \KELVIN}{40 \KELVIN}
                \MC 233.15 \KELVIN \lt \TEMP \lt 273.15 \KELVIN \NR
            \NC 1 \MC 273.15 \KELVIN \le \TEMP \NR
        \stopcases
    \stopformula

    Ice content is thrown away. (REFRENCE)

    \placefigure[top][fig:cosmo7_prior]
            {A COSMO7 prior distribution of temperature (left) and specific
            humidity (right) for the lower half of the troposphere, valid at
            2015-02-06 03:02 UTC. The means (black lines) are interpolated from
            lead times +00 and +06 of the 2015-02-06 00:00 UTC run. The grey
            shadings show the one (dark) and two (light) standard deviation
            intervals obtained by sampling.
            }
            {\externalfigure[cosmo7_prior][width=\textwidth]}

\stopsection


\startsection[title={Nordkette Slope Measurements}]

    Data by ZAMG, previously used by Daniel Meyer, describe location of
    stations and their height. Will be used as additional regressors.
    Just as surface measurements are questionable in their use as additional
    information the same can be said about these measurements which are also
    taken in the surface layer of the slope.

\stopsection

