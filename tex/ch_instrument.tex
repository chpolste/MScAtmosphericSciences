The discussed optimal estimation and linear regression methods are applied
to retrievals of temperature and humidity profiles based on data collected
in Innsbruck. A radiosonde climatology provides reference profiles for
the evaluation of synthetic performance of the retrieval techniques based
on simulated brightness temperatures and for an application to actual
radiometer data. Additional prior information is obtained from vertical
profiles forcasted by a numerical weather prediction model. Figure
\in[fig:map_ibk] gives an overview of the Innsbruck region and the locations of
all data sources.


\startsection[title={Retrieval Setup}]

    Regression models for temperature use V band channels, surface pressure
    and surface temperature as regressors.
    Regression models for humidity use K band channels, surface pressure
    and surface humidity as regressors.
    Regularization found to be important to avoid overfitting.

    MWRTM used as reference TB data set. But 0.5 K std random noise added to
    all channels before any retrievals to recreate instrument noise.

    Model covariance from MWRTM/FAP vs
    ROSENKRANZ comparison with additional 0.5 K uncorrelated instrument
    error.

    Levenberg-Marquardt method as given by Rodgers equation 5.36:

    \startformula
        ((1 + \gamma) \COVMATA^{-1} + \ITER{\FWDJAC}{i}^\top \COVMATERR^{-1} \ITER{\FWDJAC}{i}) \,
        (\ITER{\MEANVEC}{i+1} - \ITER{\MEANVEC}{i})
            = \ITER{\FWDJAC}{i}^\top \COVMATERR^{-1} (\VECY - \FWD(\ITER{\MEANVEC}{i}) - \MEANVECERR)
                - \COVMATA^{-1} (\ITER{\MEANVEC}{i} - \MEANVECA)
    \stopformula

    $\gamma$ is initially set to 3000 and is halved if the cost function
    decreased in an iteration and multiplied by 5 if the cost function
    increased. This is not the ideal way to update...

    Convergence determined with the cost function too. If the cost function
    does not decrease by more than 2 \% over 3 iterations, the iteration with
    the smallest cost function value is returned as the retrieval result. If no
    convergence is reached after 20 iterations, the iteration with the smallest
    cost function value is still returned.

\stopsection


\placefigure[top][fig:map_ibk]
    {Data locations in the Inn-valley around Innsbruck. 1: Innsbruck
    Airport (577 m), radiosonde launch location. 2: HATPRO location with
    approximate elevation scanning direction (612 m). 3: COSMO-7 model grid
    point. Map source: http://www.basemap.at}
    {\externalfigure[map][width=\textwidth]}


\startsection[title={Radiosonde Climatology},reference=ch:rasoclim]

    Radiosondes are launched regularly once per day from Innsbruck airport
    at the location of marker 1 in Figure \in[fig:map_ibk]. Launch time is
    usually between 01:00 and 04:00 UTC, which corresponds to 02:00 to 05:00
    local time in winter and 03:00 to 06:00 local time during daylight saving
    time. Here, the climatological dataset of temperature, humidity and
    pressure spanning the years 1999 to 2005 and 2009 to 2012, previously used
    by \cite[Massaro2013] and \cite[Meyer2016], is extended with profiles
    obtained between 2013 and 2016.  In total, data from 3561 radiosonde are
    available which are separated into a training data set of 3296 profiles and
    a test data set of 265 profiles which contains all data between February
    2015 and January 2016. This partitioning results in the largest overlap of
    test data with other data sources.

    Because liquid water content is not measured by any instrument during
    a balloon sounding, cloud water content is calculated by the empirical
    cloud model of \cite[Karstens1994] based on humidity and temperature
    information. This model is frequently used for microwave radiometer
    retrieval studies, e.g. by \cite[Lohnert2012,Martinet2015].

    The profiles of pressure, temperature, humidity and liquid water content
    are linearly interpolated to the 50 levels of the retrieval grid. Figure
    \in[fig:raso_prior] shows the climatology's mean profile of temperature and
    total water content together with intervals of ±1 and ±2 standard
    deviation. Temperature variability is almost constant throughout the
    troposphere, while water variability is greater near the surface and
    vanishes at the mean height of the tropopause. The latter is due to the
    small water vapor content in the upper atmosphere. Total water content
    uncertainty intervals are asymmetric because the underlying distribution is
    a Gaussian in logarithmic space (see section \in[ch:statevector]).

    Whenever the accuracy of a retrieval method is evaluated in subsequent
    sections, these radiosonde data are the reference that retrieved profiles
    are compared to. It must be noted that radiosondes provide point
    measurements, are affected by horizontal drift and take a substantial amout
    of time to reach their final height. These properties of balloon-bourne
    sensors raise concerns regarding the representativity of the measured
    vertical profiles for a fixed location. Also, only 4 \% of all radiosonde
    profiles were obtained during day time which has to be condsidered when
    assessing the applicability of retrieval methods based on the climatology.
    Inaccuracies due to these issues are however tolerated due to the lack of
    a better alternative.

    \placefigure[top][fig:raso_prior]
            {The radiosonde climatology of temperature (left) and total water
            content (vapor and liquid water) derived from soundings launched
            at Innsbruck airport. The black lines indicate the climatological
            mean values, grey shadings mark the one (dark) and two (light)
            standard deviation interals obtained from sampling the
            distribution.
            }
            {\externalfigure[raso_prior][width=\textwidth]}

\stopsection


\startsection[title=COSMO-7 Simulated Soundings]

    COSMO-7 is a regional weather prediction model for western and central
    Europe operated by MeteoSwiss. Its terrain-following grid has a
    horizontal mesh size of 6.6 km and 60 vertical levels up to 21453
    m altitude. Vertical profiles of temperature, humidity and cloud water
    content for a gridpoint near Innsbruck (marker 3 in Figure
    \in[fig:map_ibk]) have been obtained for the timespan 2015-02-06 to
    2016-01-20 from forecasts initiated at 00 UTC. Liquid water content
    $\QLIQ$ is extracted from the total cloud water content $\QCLOUD$ based on
    the temperature $T$ of the cloud according to

    \startformula
        \QLIQ = \QCLOUD \startcases
            \NC 0 \MC 233.15 \, \KELVIN \le \TEMP \NR
            \NC \frac{\TEMP - 233.15 \, \KELVIN}{40 \, \KELVIN}
                \MC 233.15 \, \KELVIN \lt \TEMP \lt 273.15 \, \KELVIN \EQSTOP \NR
            \NC 1 \MC 273.15 \, \KELVIN \le \TEMP \NR
        \stopcases
    \stopformula

    Forecasts are availabe 6 h steps and up to +30 h. Profiles are linearly
    interpolated to match any given time of retrieval between two lead times.

    Because the ground at the gridpoint near Innsbruck is located on the model
    grid more than 400 m too high at an altitude of 1093 m, the COSMO-7
    profiles have to be extrapolated to the true ground height. At first it was
    considered to simply connect the lowest level of the COSMO-7 profile to
    actual surface measurements of temperature and humidity. This however
    disregards information from the boundary layer processes resolved by the
    model which would then be elevated features. Instead, profiles are
    stretched to the ground by replacing the original height grid of COSMO-7
    by one that ends at the same altitude but starts at the true surface.
    The new grid is chosen such that lower levels are streched more than
    upper levels but the overall relative spacing of the model levels is
    approximately preserved. Because this procedure does not take temperature
    gradients into account the resulting profiles likely have a significant
    bias relative to the true atmospheric state. To counter this, the mean
    bias of the model profiles with respect to the radiosondes from Innsbruck
    airport is removed from the dataset. Because there are only 253 times
    where a radiosounding exists parallel to a COSMO-7 forecast this bias is
    determined from the entire data set. An important implication of this
    choice is that COSMO-7 profiles used for the retrieval are not independent
    of the radiosonde reference data set. This inconvenience is accepted in
    favor of a more robust bias determination based on a larger data set.

    In order to use the COSMO-7 forecasts as a prior for optimal estimation
    retrievals a covariance matrix quantifying the model's uncertainty must
    be known. As for the bias, this matrix is calculated from a comparison
    with the reference radiosonde profiles. Figure \in[fig:cosmo7_prior]
    shows the uncertainty range of a model profile attached to a forecast
    interpolated to 2015-09-11 03:48 UTC in the lower half of the troposphere.
    Temperature uncertainty is highest in the lower troposphere, but smaller
    outside the boundary layer. Uncertainty increases above 6 km only in the
    tropopause region. The water content uncertainty is approximately constant
    up to 5 km above which it is decrasing toward zero. The example profile
    indicates that COSMO-7 is able to simulate temperature inversions in the
    lower atmosphere.

    \placefigure[top][fig:cosmo7_prior]
            {A COSMO-7 prior distribution of temperature (left) and total water
            content humidity (right) valid at 2015-02-06 03:02 UTC. The means
            (black lines) are interpolated from lead times +00 and +06 of the
            2015-02-06 00:00 UTC run. The grey shadings show the one (dark) and
            two (light) standard deviation intervals obtained by sampling.
            Shown is the lower half of the troposphere.
            }
            {\externalfigure[cosmo7_prior][width=\textwidth]}

\stopsection


\startsection[title=HATPRO Observations,reference=ch:hatpro]

    The microwave radiometer operated by the The Institute of Atmospheric and
    Cryospheric Sciences at the University of Innsbruck is a Humidity and
    Temperature Profiler (HATPRO), built by the Radiometer Physics GmbH
    \cite[authoryears][Rose2005]. It records brightness temperatures in 14
    channels with a temporal resolution of 1 s. Channels in the K band
    correspond to frequencies of 22.24, 23.04, 23.84, 25.44, 26.24, 27.84 and
    31.40 GHz. Channels in the V band correspond to frequencies of 51.26,
    52.28, 53.86, 54.94, 56.66, 57.30 and 58.00 GHz. These are marked with
    arrows in Figure \in[fig:absorption]. The receiver antenna can be rotated
    to perform elevation scans. Environmental pressure, temperature and
    humidity are additionaly measured by the instrument and it has
    a precipitation sensor which is translated to a rain/no rain flag in the
    output data.

    The radiometer is set up on the roof of a university building at an
    altitude of 612 m above sea level. Figure \in[fig:map_ibk] shows the
    location of the radiometer site in the Inn Valley (marker 2). The distance
    to the radiosonde launch site at Innsbruck airport (marker 1) is
    approximately 2.5 km. The height difference between these sites is 35 m
    with the HATPRO located higher. The elevation scan direction (indicated by
    the triangle) is downvalley, in the opposite direction of the airport.

    Data from the instrument are available in the timespan from August to
    October 2015. In this period, HATPRO was configured to perform boundary
    layer scans every 10 minutes with averaging times of approximately 30 s
    each at zenith angles of 0°, 60°, 70.8°, 75.6°, 78.6°, 81.6°, 83.4°.
    HATPRO observations were matched to 26 radiosonde launches by taking the
    closest boundary layer scan with at most 30 minutes time difference to the
    launch of the radiosonde.

    Bias (wrt to numerical model) correction by comparison with radiosonde
    climatology. This means that measurements are not independent of referenc
    data anymore but else too little data are available. In operational
    applications one might do rolling updates to the bias correction as well.
    Profits non-verifiable data during the day.

\stopsection

