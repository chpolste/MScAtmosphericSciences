Formulate retrieval problem.

Introduce optimal estimation and regression mention Kernel Methods, ... as
alternative regression methods.

\startsection[title=Bayesian Statistics]

    The Bayesian view of statistics is ideal for the formulation of retrieval
    problems. 

    Conditional probability, diachronic interpreation, subjective priors,
    making assumptions explicit, focus on pdf from which values can be
    extracted due to different criteria, pdf carries uncertainty estimates, ...

    Central role: Bayes' theorem.
    
    \placeformula[eq_ret_bayes]
    \startformula
        P(x|y) = \frac{P(y|x) P(x)}{P(y)}
    \stopformula

    $P(x|y)$ likelihood

    $P(y|x)$ posterior

    $P(x)$ prior

    $P(y)$ normalization, can be obtained by integrating $P(y|x)P(x)$
    over all possible states $x$.

\stopsection

\startsection[title=The Multivariate Gaussian Distribution]

    Bayesian view relies heavily on probablility density functions. In a
    theoretical framework it is desireable to work with pdfs that allow
    analytic solutions for a given problem. The most commonly used pdf is the
    Gaussian distribution:

    \placeformula[ret_gauss]
    \startformula
        \GAUSS{\VECX}{\MEANVEC}{\COVMAT}
        = \frac{1}{(2 \pi)^{N/2} \det(\COVMAT)^{1/2}}
          \exp \left( -\frac{1}{2}(\VECX - \MEANVEC)^\top \COVMAT^{-1} (\VECX - \MEANVEC) \right)
    \stopformula

    where $\VECX, \MEANVEC \in \REALS^N$ and $\COVMAT \in \REALS^{N
    \times N}$ and the exponenial function is applied component-wise.

    The assumption that a continuous variable follows a Gaussian distribution
    is not only popular due to its well understood analytic behavior. central
    limit theorem, maximum entropy.

    The mean vector $\MEANVEC$ of the Gaussian distribution is the
    location of its maximum. The covariance matrix $\COVMAT$ determines
    the shape of the distribution. For a proper Gaussian distribution, it is
    always a symmetric positive definite matrix. When all off-diagonal elements
    of $\COVMAT$ are zero, each component of the multidimensional
    distribution is independent from the others. If nonzero off-diagonal
    elements exist, some or all componentes are correlated.

    Transformation in eigenspace.

    Properties: Product of Gaussian distributions Maximum likelihood
    estimators. Plot stuff.

    cite Bishop, Rodgers

\stopsection


\startsection[title=Optimal Estimation]

    Reference Rodgers, explain how it fits together.

    The optimal estimation approach directly uses Bayes' theorem
    \ineq{eq_ret_bayes} to obtain the posterior probability distribution
    $\POSTERIOR$ using a forward model and prior information of the state
    vector.

    Call it physical retrieval because it uses the forward model.

    The assumptions for the prior {\PDF} and the likelihood function
    are that they are Gaussian.

    Justify assumptions.

    The prior $\PRIOR$ should contain all knowledge about the state vector
    before the measurement is evaluated. This knowledge might come from
    a climatology, output from a NWP model, information from other instruments
    and any combination of these sources. The construction of an optimal prior
    is not an easy task and will be discussed in section \in[ret_constprior].
    In any case, due to the assumption that the {\PDF} is Gaussian it will have
    the form:

    \placeformula[eq_ret_optprior]
    \startformula
        \PRIOR = \GAUSS{\VECX}{\MEANVECA}{\COVMATA}
    \stopformula

    The subscript stands for \quotation{a-priori}.

    The likelihood function $\LIKELIHOOD$ expresses the confidence that the
    observation $\VECY$ result of a state $\VECX$. This confidence is
    quantified with the help of the forward model, which can be formulated as:

    \placeformula[eq_ret_fwdmodel]
    \startformula
        \VECY = \FWD(\VECX) + \ERR
    \stopformula

    Where $\FWD$ is the forward model operator and $\ERR$ is an error term,
    arising from measurement noise, inaccuracies of the forward model,
    interpolation errors and possibly other terms. Again, the assumption is
    that the errors are Gaussian, centered around a systematic bias
    $\MEANVECERR$ and with a convariance matrix $\COVMATERR$:

    \placeformula[eq_ret_opterrpdf]
    \startformula
        P(\ERR) = \GAUSS{\ERR}{\MEANVECERR}{\COVMATERR}
    \stopformula

    A discussion of how $\MEANVECERR$ and $\COVMATERR$ can be determined is
    given in section \in[rtm_errors]. Substituting \ineq{eq_ret_fwdmodel}
    into \ineq{eq_ret_opterrpdf} and changing the independent variable of
    the {\PDF} results in the likelihood function:
    
    \placesubformula
    \startformula
    \startalign[n=3,align={right,middle,left}]
        \NC P(\VECY - \FWD(\VECX)) = \NC \GAUSS{\VECY - \FWD(\VECX)}{\MEANVECERR}{\COVMATERR} \NC \NR
        \NC = \NC \GAUSS{\VECY}{\FWD(\VECX)+\MEANVECERR}{\COVMATERR} \NC = \LIKELIHOOD \NR[eq_ret_optlikelihood][]
    \stopalign
    \stopformula

    Knowing \ineq{eq_ret_optprior} and \ineq{eq_ret_optlikelihood} the
    posterior {\PDF} is given by:

    \placeformula[eq_ret_optposterior]
    \startformula
        \POSTERIOR
        = \frac{\LIKELIHOOD \PRIOR}{\NORMALIZATION}
        = \frac{\GAUSS{\VECY}{\FWD(\VECX)+\MEANVECERR}{\COVMATERR}~\GAUSS{\VECX}{\MEANVECA}{\COVMATA}}{\NORMALIZATION}
    \stopformula

    \startsubsection[title=The Maximum A-Posteriori Solution]

        Gaussian Error Assumption, linearize forward model → jacobian/adjoint,
        analytic solution for cost function minimization.

        Where does 1D-VAR come from?

    \stopsubsection

    \startsubsection[title={Constructing the Prior},reference=ret_constprior]

        Choosing the prior (=background) state. Constructing the covariance
        matrix.

    \stopsubsection

    \startsubsection[title=A Generalization to Multiple Measurements]

        As shown by \cite[Rodgers2000] in chapter 4.1.1, there is a more
        general way to construct the posterior {\PDF} $\POSTERIOR$. If $n$
        measurements $\VECY_i$ are available that can be related to the state
        vector by forward models $\FWD_i$ and the forward model errors are
        pairwise independent, then Bayes' theorem takes the form

        \startformula
        \startalign[n=2,align={right,left}]
            \NC P(\VECX|\VECY_1,\dots,\VECY_n) =
                \NC \frac{P(\VECY_1,\dots,\VECY_n|\VECX)\PRIOR}{P(\VECY_1,\dots,\VECY_n)} \NR
            \NC = \NC \frac{\PRIOR}{P(\VECY_1,\dots,\VECY_n)} \prod_{i=1}^n P(\VECY_i|x) \NR
            \NC = \NC \frac{\PRIOR}{P(\VECY_1,\dots,\VECY_n)}
                \prod_{i=1}^n \GAUSS{\VECY_i}{\FWD_i(\VECX) + \MEANVEC_i}{\COVMAT_i} \EQCOMMA \NR
        \stopalign
        \stopformula

        where the independence property of the forward model errors has been
        used for the second identity. With this formulation, any measurements
        that give information on the state vector can be included in the
        optimal estimation procedure. This includes the a-priori knowledge,
        which might be interpreted as a \quotation{virtual measurement} having
        the identity function as a forward model and the prior state
        uncertainties as errors. In this interpretation, no prior knowledge
        is explicitly included in the model, i.e. $\PRIOR$ is constant, and the
        maximum a-posteriori solution is equivalent to a maximum likelihood
        solution.

        Is this used in the thesis? % TODO
        
    \stopsubsection

    \startsubsection[title=Introducing Constraints by Regularization]

        Introduce penalties to cost function. Advantage: Simpler to set up
        than modifying prior, but no representation of additional knowledge
        in posterior.

        Example: \cite[Hewison2006], penalizing high LWC and references
        therein.

    \stopsubsection

\stopsection


\startsection[title=Linear Regression]

    Based on climatology, where to get climatology from.

    \startsubsection[title=Quantifying Uncertainty]

    \stopsubsection

    \startsubsection[title=Including Additional Information]

    \stopsubsection
    
\stopsection


\startsection[title=Neural Network Regression]

        

\stopsection


\startsection[title=Comparison of Techniques]

    Ease of use, computational efficiency, inclusion of additional information.

\stopsection

