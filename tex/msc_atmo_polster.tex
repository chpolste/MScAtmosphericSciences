\usecolors[xwi]
\usemodule[chart]

\define\DocTitleFooter{Bayesian Retrieval of Thermodynamic Atmospheric Profiles from Ground-based MWR Data}
\define\Author{Christopher Polster}
\define\DateOfCompletion{2016-08-12} % TODO

\input mycommands

\mainlanguage[en]
\hyphenation{}

% DIN-A4 Paper
\setuppapersize[A4]
\setuplayout[
        backspace=35mm,
        width=150mm,
        header=0mm,
        footer=0mm
        ]

% Create links in Table of Contents and set pdf metadata
\setupinteraction[
        state=start,
        color=black,
        contrastcolor=black,
        style=,
        focus=standard,
        title={Bayesian Retrieval of Thermodynamic Atmospheric Profiles from Ground-based Microwave Radiometer Data},
        subtitle={Masters's Thesis, 2016},
        author={Christopher Polster},
        ]
\placebookmarks[chapter,section,subsection]

% Titles for TOC and Refs
\setupheadtext[
        content={Table of Contents},
        pubs=References
        ]

% Fonts
% Text uses serif font
\setupbodyfont[11pt,serif]
\definebodyfontenvironment[11pt][a=12pt,b=13pt,c=14pt,d=20pt]

% Headings etc. are sans-serif
\definefontfamily[titlefont][sans][dejavusans]
\definefont[CoverHeadingFont][dejavusans at 19pt]
\definefont[AbstractHeadingFont][dejavusans at 11pt]
\definefont[ChapterFont][dejavusans at 19pt]
\definefont[SectionFont][dejavusans at 13pt]
\definefont[SubsectionFont][dejavusans at 12pt]
\definefont[TOCHeadingFont][dejavusans at 11pt][2]
\definefont[PageNumberFont][dejavusans at 10pt]
\definefont[FooterFont][dejavusans at 8pt]
\definefont[FigureFont][dejavusans at 8.5pt][1.1] % the 1.1 is line height
\definefont[FigureCaptionFont][dejavusansbold at 8.5pt][1.1]

% Styles for headings etc.
\definealternativestyle[FigureStyle][\FigureFont]
\definealternativestyle[FigureCaptionStyle][\FigureCaptionFont]
\definealternativestyle[TOCStyle][\TOCHeadingFont]
\definealternativestyle[ChapterStyle][\ChapterFont]
\definealternativestyle[SectionStyle][\SectionFont]
\definealternativestyle[SubsectionStyle][\SubsectionFont]

% Don't reset section numbering in new chapter
\definestructureresetset[nosecreset][1,1,0][1] % [part, chapter, section][default]
\setuphead[sectionresetset=nosecreset]
% No chapter numbering, apply heading styles
\setuphead[chapter][number=no,style=ChapterStyle,page=yes]
\setuphead[section][style=SectionStyle,sectionsegments=section]
\setuphead[subsection][style=SubsectionStyle,sectionsegments=section:subsection]

% Footnotes
\define[1]\footnotebrack{\narrownobreakspace\high{[#1]}}
\setupnotation[footnote][alternative=text]
\setupfootnotes[
        way=bytext,
        frameoffset=0mm,
        topframe=on,
        rule=off,
        toffset=1mm,
        roffset=-14cm,
        before={\blank[7mm]}%,
        %textcommand=\footnotebrack
        ] % fixes weird spacing when only one line of footnote

% Table of contents
\setuplist[chapter][style=TOCStyle]
\setupcombinedlist[section,subsection][alternative=c]
\setuplist[section][width=9mm]
\setuplist[subsection][width=9mm,margin=9mm]

% Figures
\setupexternalfigures[directory=figures/]
\setupcaptions[figure][
        style={FigureStyle},
        suffix={:},
        headstyle={FigureCaptionStyle},
        prefixsegments=none,
        width=fit,
        way=bytext,
        spaceafter=2mm
        ]

% Formulas
\defineseparatorset[none][][]
\setupformulas[way=bytext,prefixsegments=none,numberseparatorset=none]
% Increase spacing around binary relation symbols (default is 5mu plus 5mu).
\thickmuskip=10mu plus 5mu

% Bibliography
\setupbibtex[database=literature,sort=author]
\setuppublications[criterium=cite,alternative=apa,sorttype=bbl,refcommand=authoryear]
\setuppublicationlist[artauthoretallimit=40,criterium=all]
\setupcite[authoryears][pubsep={; },lastpubsep={; },inbetween={ },compress=no]

% Taken from bibl-apa.tex and added doi
\setuppublicationlayout[article]{%
   \insertartauthors{}{ }{\insertthekey{}{ }{}}%
   \insertpubyear{(}{). }{\unskip.}%
   \insertarttitle{\bgroup }{\egroup. }{}%
   \insertjournal{\bgroup \it}{\egroup}
    {\insertcrossref{In }{}{}}%
   \insertvolume
    {\bgroup \it, }
    {\egroup\insertissue{\/(}{)}{}\insertpages{, }{.}{.}}
    {\insertpages{, pp. }{.}{.}}%
   \insertdoi{ doi:}{.}{}%
   \insertnote{ }{.}{}%
   \insertcomment{}{.}{}%
}


% Section block setups
% This could probably be solved more elegantly, but it worked for my
% bachelor's thesis and I don't see the point in redoing something that works.

% Cover
\definesectionblock[CoverSectionBlock][,]
\setupsectionblock[CoverSectionBlock][page=right]
\startsectionblockenvironment[CoverSectionBlock]
    \switchtobodyfont[titlefont,10pt]
    \setupinterlinespace[line=14pt]
    \setupwhitespace[medium]
    \raggedcenter
\stopsectionblockenvironment

% Table of Contents
\definesectionblock[TOCSectionBlock][,]
\setupsectionblock[TOCSectionBlock][page=right]
\startsectionblockenvironment[TOCSectionBlock]
    \setupwhitespace[0.5em]
    \setuppagenumbering[state=stop,alternative=doublesided]
\stopsectionblockenvironment

% Frontmatter: Acknowledgements, Preface
\setupsectionblock[frontpart][page=right]
\startsectionblockenvironment[frontpart]
    \setuplayout[footerdistance=8mm,footer=6mm]
    \setcounter[userpage][1] % Reset page counter
    \setuppagenumbering[
            state=start,
            location={footer,right},
            left={},
            right={},
            alternative=doublesided,
            style=\PageNumberFont
            ]
    \setupuserpagenumber[numberconversion=romannumerals]
    \setupwhitespace[medium]
    \setupinterlinespace[line=1.5em]
\stopsectionblockenvironment

\definesectionblock[AbstractSectionBlock][,]
\setupsectionblock[AbstractSectionBlock][page=right]
\startsectionblockenvironment[AbstractSectionBlock]
    \setuppagenumbering[state=stop,alternative=doublesided]
    \setupinterlinespace[line=1.5em]
    \setupwhitespace[medium]
    \setupnarrower[middle=10mm]
    \page[right]
\stopsectionblockenvironment

% Bodymatter: all text
\setupsectionblock[bodypart][page=no]
\startsectionblockenvironment[bodypart]
    \setuplayout[footerdistance=8mm,footer=6mm]
    \setcounter[userpage][1] % Reset page counter
    \setuppagenumbering[
            state=start,
            location={footer,right},
            left={},
            right={},
            alternative=doublesided,
            style=\PageNumberFont
            ]
    \setupfootertexts[{\FooterFont {\DocTitleFooter}}]
                     [pagenumber]
                     [pagenumber]
                     [{\FooterFont Master's Thesis of {\Author} (2016)}]
    \setupbackgrounds[footer][text][topframe=on]
    \setupwhitespace[medium]
    \setupinterlinespace[line=1.5em]
\stopsectionblockenvironment


% Backmatter: References
\setupsectionblock[backpart][page=no]
\startsectionblockenvironment[backpart]
    \setuppagenumbering[alternative=doublesided,style=\PageNumberFont]
    \setuplayout[footerdistance=8mm,footer=6mm]
    \setupfootertexts[{\FooterFont {\DocTitleFooter}}]
                     [pagenumber]
                     [pagenumber]
                     [{\FooterFont Master's Thesis of {\Author} (2016)}]
    \setupbackgrounds[footer][text][topframe=on]
    \setupwhitespace[medium]
    \setupinterlinespace[line=1.5em]
\stopsectionblockenvironment


\starttext

\startsectionblock[CoverSectionBlock]

    \dontleavehmode
    \blank[10mm]
    \dontleavehmode
    \externalfigure[uibk_logo.pdf][height=25mm,location=middle]
    \enskip \enskip \enskip \enskip \enskip \enskip \enskip
    \externalfigure[acinn_logo.pdf][height=25mm,location=middle]
    \blank[13mm]
    \dontleavehmode
    \blackrule[width=8cm,height=0.5mm]
    \blank[22mm]

    {\CoverHeadingFont \strut 
    Bayesian \thinspace Retrieval \thinspace of \thinspace Thermodynamic\blank[6mm]
    Atmospheric \thinspace Profiles \thinspace from \thinspace Ground-based \blank[6mm]
    Microwave \thinspace Radiometer \thinspace Data
    \strut}

    %{\CoverHeadingFont \strut \DocSubtitle \strut}

    \blank[22mm]

    \dontleavehmode
    \blackrule[width=8cm,height=0.5mm]

    \blank[13mm]

    {\bold Master's Thesis in Atmospheric Sciences}

    \blank[10mm]

    Submitted to the Faculty of Geo- and Atmospheric Sciences \break
    in Partial Fulfillment of the Requirements for the Degree of Master of Science

    \blank[5mm]

    by {\bold \Author}

    \blank[15mm]

    University of Innsbruck, \DateOfCompletion

    \blank[15mm]

    Advisor: Prof. Dr. Mathias Rotach \break

\stopsectionblock

\startsectionblock[AbstractSectionBlock]

    \dontleavehmode
    \blank[10mm]
    \startnarrower
        {\AbstractHeadingFont Abstract}
        
        Ground-based microwave radiometers are increasingly used to retrieve
        vertical temperature, humidity and cloud information of the atmosphere.
        Such information is valuable for boundary layer research, weather
        forecasting and experiments have been undertaken to assimilate
        radiometer observations into numerical weather prediction models.
        Multiple methods exist to perform the retrieval, differing in their
        data requirements, ease of use and flexibility to include measurements
        from sensors other than the radiometer.

        A linear regression and an optimal estimation technique have been
        derived and implemented in this thesis. Important properties of these
        methods are discussed and their accuracy is evaluated with data from
        radiosoundings and radiometer measurements in Innsbruck. Standard
        deviations of temperature retrievals from an optimal estimation scheme
        integrating forecasts from a numerical weather prediction model
        are found be be less than 1.2 K throughout the troposphere relative to
        reference measurements from radiosoundings. The least accurate region
        is located between 1.5 and 3 km above ground level. There the numerical
        forecasts are not as accurate as in the upper troposphere and the
        information content of the radiometer has already decreased
        substantially compared to the lower atmosphere.

        In two case studies it is found that the optimal estimation scheme
        is promising for the retrieval of temperature inversions which have
        been an often studied problem of microwave radiometer retrieval. The
        quality of a-priori information, particularly its capability of
        providing a first guess of the features that an atmospheric state
        exhibits is, found to be a major influence on the retrieved vertical
        profiles.

        Also presented in this thesis is a prototype of a numerical radiative
        transfer model for the microwave region. It is a minimalistic
        implementation in a high-level programming language and able
        to calculate linearizations of itself by automatic differentiation.
        The model is found to be sufficiently accurate for use in retrieval
        applications.

        \stopnarrower


\stopsectionblock

\startsectionblock[TOCSectionBlock]

    \completecontent

\stopsectionblock

\startfrontmatter

    \startchapter[title=Acknowledgements]
        \input ch_acknowledgements
    \stoptitle

    \page[even,empty] % TODO: take care of this when all is written

    \startchapter[title=Preface]
        \input ch_preface
    \stoptitle

    %\page[even,empty]

\stopfrontmatter

\startbodymatter
    
    \input flowcharts

    \startchapter[title=Introduction]
        \input ch_introduction
    \stopchapter

    \startchapter[title=Retrieval Techniques]
        \input ch_retrieval
    \stopchapter

    \startchapter[title=Radiative Transfer and State Representation]
        \input ch_radiative_transfer
    \stopchapter

    \startchapter[title={Data and Methodology}]
        \input ch_data
    \stopchapter

    \startchapter[title=Retrieval Results and Discussion]
        \input ch_results
    \stopchapter

    \startchapter[title=Conclusions and Outlook]
        \input ch_conclusions
    \stopchapter

\stopbodymatter

\startbackmatter

    \startchapter[title=References]

        All online references were last accessed 2016-08-07.
        \blank[1em]

        \placepublications

    \stopchapter

    \page[left] %TODO

\stopbackmatter

\startsectionblock[AbstractSectionBlock]

    \dontleavehmode
    \startsubject[title=Eidesstattliche Erklärung]
    Ich erkläre hiermit an Eides statt durch meine eigenhändige Unterschrift,
    dass ich die vorliegende Arbeit selbständig verfasst und keine anderen als
    die angegebenen Quellen und Hilfsmittel verwendet habe. Alle Stellen, die
    wörtlich oder inhaltlich den angegebenen Quellen entnommen wurden, sind als
    solche kenntlich gemacht.

    \setuplines[align=paragraph]

    Die vorliegende Arbeit wurde bisher in gleicher oder ähnlicher Form noch nicht als\break
    Magister-/Master-/Diplomarbeit/Dissertation eingereicht. 

    \dontleavehmode
    \blank[1mm]

    Innsbruck, \DateOfCompletion

    \blank[20mm]

    \blackrule[width=9cm,height=0.2mm]
    \hbox{\raise2mm\hbox{\small Christopher Polster}}

    \stopsubject

\stopsectionblock

\stoptext
