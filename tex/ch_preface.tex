Hardest part of writing about research for me is the presentation of
unfulfilled expectations and failed experiments. Current publication system
results in one being used to seeing positive results almost exclusively. This
brings a certain naivety. In reality most things are (much) harder than they
look. But research is a process in time. One starts with a set of knowledge and
expectations and both are in continuous change due to the research being done.
While thinking about this I realized a nice connection to the theme of this
thesis: one starts with knowledge and expectations based on this knowledge.
During the research this knowledge is updated with experience and results. The
research is dependent on the expectations which govern the conducted
experiments.  After having conducted the research one's knowledge has been
updated with the results.  Representing research with $r$ and knowledge with
$k$ this process is described exactly by Bayes' theorem:

\startformula
    \PROB{k \GIVEN r} = \frac{\PROB{r \GIVEN k} \, \PROB{k}}{\PROB{r}}
\stopformula

Which parts to write about, only give final state as usually done in scientific
literature or take a more exploratory approach describing the results and the
process including things that are unclear or do not work (reference
http://jvns.ca/?). Both have merits and flaws.

The original idea of this thesis was to study neural network retrievals as an
alternative to linear regression models. During the literature research it
became apparent to me that the optimal estimation approach is a mathematically
much more elegant approach to the problem and coincidentally also provides
good retrieval performance. I therefore changed the focus and started working
on an implementation of this technique. NN training w/ autodiff → idea for a
custom RTM. Realization that radiative transfer is not as easy as it often
seems in the retrieval literature. My expectations regarding the Bayesian
approach to linear regression were damped by the realization that the
predictive distribution did not represent the "right" kind of error.

Despite their (partial) failures, the developed numerical model and the theory
of Bayesian linear regression are included in this thesis as they represent
a significant part of the work and might still be useful for following research
if only as a reminder of what has not worked in the past.

\cite[Blumberg2015] used MonoRTM version 4.2
at different elevation angles. It is unclear how they managed this.

