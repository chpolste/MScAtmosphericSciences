This thesis was originally about the implementation of a neural network
technique for the retrieval of temperature and humidity profiles but in the end
it turned into an implementation of the optimal estimation method. This change
of mind occurred during the initial literature review phase where I began to
understand the limitations of regression models and started gaining more
appreciation for the Bayesian approach that optimal estimation is based on.
I have therefore chosen to present the theoretical work of this thesis, including
the derivation of linear regression, from a Bayesian standpoint.

The implementation of a radiative transfer model was not part of the original
plan for this work but the desire for an easy to use model arose after some
frustrations with the rather antiquated user interface of an existing model.
The simple but powerful idea of automatic differentiation had fascinated me
since I first read about it in the context of neural network training, so
I decided to use it in my model. Judging from the short presentation of
radiative transfer in most publications, I naively assumed that the
implementation of a numerical model would be a rather simple task and in terms
of code it really is not very complicated.  But I soon found that the devil is
in the details.  Getting the model right took a substantial amount of time,
much more than I would have needed to set up an existing solution, but
I learned a lot from the experience and that justifies the effort for me
personally.

This actually seems to be the theme of this work: reinventing the wheel in
order to get to know all aspects of the retrieval problem in depth. While this
is not best scientific practice, I think a Master's thesis provides a good
opportunity to take such an approach without wasting much of anyone's time
(except possibly mine).

There is much more literature on retrievals and radiative transfer than
referenced here. The plentiful citations in the text should provide a decent
point of origin for anyone trying to find more information. A guiding
influence that helped me orient myself in the field of microwave radiometer
retrievals was the PhD thesis of Tim \cite[Hewison2006] in which he describes
the setup of an optimal estimation scheme for the retrieval of temperature and
humidity from ground-based microwave radiometer observations with great detail.

