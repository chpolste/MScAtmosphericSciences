This thesis was originally about the implementation of a neural network
technique for the retrieval of temperature and humidity as an alternative to
the linear regression method which has been used in the past at the Institute
for Atmospheric and Cryospheric Sciences in Innsbruck. In the end it turned
into an implementation of the optimal estimation method. This change of topic
occured during the initial literature review. I believe that despite its
complexity this technique provides the best approach to the retrieval problem
currently known and hope to argue in favor of this belief with the following
discussions and results.  I have chosen to present all derivations in the
Bayesian framework after seeing how universally Bayes' theorem can be applied
to the construction of machine learning models and estimation problems in the
books by \cite[Bishop2006] and \cite[Downey2013]. Initially it was not planned
to also implement a radiative transfer model, but after some frustration with
the interface of MonoRTM, I decided to write a model able to provide an exact
linearization of itself. I have been fascinated by the simple but powerful idea
of automatic differentiation since I first read about it in the context of
neural network training, so I decided to use it in my model. Judging from the
short presentation of radiative transfer in most publications, I naively
assumed that the implementation of a numerical model would be a rather simple
task but I soon found that the devil is in the details. Getting the model right
took a substantial amout of time, much more than I would have needed to set up
a mature solution like ARTS, but I learned a lot in that time and that
justifies the effort for me personally.  That actually seems to be the theme of
this work: reinventing the wheel in order to get to know all aspects of the
retrieval problem in depth.  While this is not best scientific practice,
I think a Master's thesis is a good place for such an approach without wasting
much of anyone's time except possibly mine.

Finally, I must mention that there is much more literature on retrievals and
radiative transfer than referenced here. The plentiful citations in the
text should however provide a decent point of origin for anyone trying to find
more information.

