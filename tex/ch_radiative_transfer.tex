The only common retrieval technique that does not depend on a numerical
radiative transfer model is measurement based regression. Time series of
simultaneous radiometer observations and radiosonde ascents to train the model
with are however rare. The ability to simulate radiometer measurements
provides a way to generate data pairs from radiosonde climatologies or NWP
forecasts. The optimal estimation approach directly incorporates a numerical
radiative transfer model (RTM) into the retrieval procedure and additionally
makes use of its linearization. The development of microwave radiative transfer
models is therefore closely linked to the development of retrieval techniques.

This chapter reviews the fundamental principles of microwave radiative transfer
in the atmosphere, discusses issues related to radiative transfer simulations
and the representation of atmospheric state for retrieval purposes and ends
with the presentation of a numerical model implementation.

\startsection[title=The Radiative Transfer Equation]

    The differential form of the radiative transfer equation in
    a non-scattering medium is given by

    \placeformula[eq:rte_general]
    \startformula
        \DERIV{I_\nu}{s} = - I_\nu \ABSCOEF + F
    \stopformula

    where $I_\nu$ is radiative power per frequency, solid angle and area in
    a specific direction (spectral radiance), $\ABSCOEF$ is the absorption
    coefficient of the medium, $F$ contains all sources of radiation and $s$ is
    distance in beam direction.
    \cite[alternative=authoryears,left={(e.g. }][Buehler2005]
    The only source term considered here is the emission of the medium itself
    in thermal equilibrium according to Plank's law

    \startformula
        F = \epsilon \, B_\nu(T)
    \stopformula

    where $\epsilon$ is the emission coefficient of the grey body and
    $B_\nu(T)$ is the Planck function describing the emitted spectral radiance
    of a black body at a given temperature $T$. The emission coefficient
    quantifies the difference of the medium to a perfect black body as
    described by the Planck function and dependent on the considered frequency.
    Because thermodynamic equilibrium is assumed Kirchhoff's law allows to
    equal the absorption and emission coefficients of the medium

    \startformula
        0 \le \ABSCOEF = \epsilon \le 1
    \stopformula

    Substituting the results of Plank's law and Kirchhoff's law into the
    radiative transfer equation \ineq{rte_general} yields

    \startformula
        \DERIV{I_\nu}{s} = - \ABSCOEF(I_\nu - B_\nu(T)) \EQSTOP
    \stopformula

    For temperatures occuring in the atmosphere and the spectral region of
    microwave radiation the Planck function $B_\nu (T)$ is approximated
    well by the Rayleigh-Jeans approximation which relates the spectral
    radiance at a given frequency linearly to temperature $T$

    \startformula
        B_\nu(T) \approx \frac{2 \nu^2 k T}{c^2}
    \stopformula

    where $c$ is the speed of light and $k$ is the Boltzmann constant.
    Microwave radiometers measure brightness temperatures instead of spectral
    radiances. Brightness temperature $\BT$ is the temperature of a black body
    that emits the same spectral radiance as an observed grey body. In the
    Rayleigh-Jeans limit the relationship between temperature and brightness
    temperature is particularly simple: $\BT = \epsilon \, T$. Expressing
    observed spectral radiance in terms of Planck emission and again
    applying the Rayleigh-Jeans approximation

    \startformula
        I_\nu := \epsilon \, B_\nu(T)
            \approx \epsilon \, \frac{2 \nu^2 k T}{c^2}
            = \frac{2 \nu^2 k}{c^2} \, \BT \EQCOMMA
    \stopformula

    and noting that the term $\frac{2 \nu^2 k}{c^2}$ is constant for a
    fixed frequency, the radiative transfer equation \ineq{rte_general} for
    a given frequency can be expressed exclusively in terms of temperature and
    brightness temperature

    \placeformula[eq:rte_radiometer]
    \startformula
        \DERIV{\BT}{s} = - \ABSCOEF (\BT - T) \EQSTOP
    \stopformula

    Because $T = T(s)$, analytical solutions of equation \ineq{rte_radiometer}
    can only be found in a few special cases and numerical solutions have to
    be seeked in general.

\stopsection


\startsection[title={A Solution for Ground-based Radiometer Applications}]

    For a radiometer looking up or at an angle to the side from the surface the
    absorbing and emitting medium is the atmosphere. To simulate a radiometer
    measurement by solving the radiative transfer equation, the atmospheric
    state has to be projected onto the light path observed by the radiometer.
    Since atmospheric state is usually known as a function of altitude the
    first step towards a solution of \ineq{rte_radiometer} is a coordinate
    transform into height coordinates

    \startformula
        \DERIV{s}{z} = - \frac{1}{\cos(\theta)} \EQSTOP
    \stopformula

    $z$ is the altitude coordinate and $\theta$ the zenith angle describing the
    deviation of the radiometers viewing direction from zenith. Because $z$
    increases with height but $s$ is the distance along the light path which
    points downward (the observed radiation is downwelling), $\DERIV{s}{z}$
    must be negative. It is convenient for the solution of
    \ineq{rte_radiometer} to introduce optical depth

    \startformula
    \startalign[n=3,align={left,right,left}]
        \NC \NC \OD(s) := \NC \int_{s_0}^{s} \ABSCOEF(u) \diff u \NR
        \NC \Rightarrow~ \NC \OD(z) = \NC
            - \frac{1}{\cos(\theta)} \int_{z_0}^{z} \ABSCOEF(u) \diff u \NR
        \NC \Rightarrow~ \NC \DERIV{\OD}{z} = \NC
            - \frac{\ABSCOEF}{\cos(\theta)} \EQSTOP \NR
    \stopalign
    \stopformula

    Applying the coordinate transformation to \ineq{rte_radiometer},
    multiplying by $e^{\OD}$ and applying the product rule of differentiation
    yields

    \startformula
        \DERIV{\BT}{z} e^\OD + \frac{\ABSCOEF \BT e^\OD}{\cos(\theta)}
        = \DERIV{(\BT e^\OD)}{z}
        = - \frac{\ABSCOEF T e^\OD}{\cos(\theta)} \EQSTOP
    \stopformula

    Integrating this differential equation from the surface ($z = 0$) to space
    ($z = \infty$) gives an expression for the brightness temperature observed
    by the (virtual) radiometer

    \placeformula[eq:rte_solution]
    \startformula
        \BT(0) = \BT(\infty) e^{\OD(\infty)} + \frac{1}{\cos(\theta)}
            \int_{0}^{\infty} \ABSCOEF(z) T(z) e^{\OD(z)} \diff z \EQSTOP
    \stopformula

    $T_B(\infty)$ is the brightness temperature of the cosmic background
    radiation which has a value of approximately 2.75 K for frequencies in the
    microwave spectrum \cite[authoryears][Westwater2004]. It is important to
    note that \ineq{rte_solution} is only valid at off-zenith angles if the
    vertical profiles of $T$ and $\ABSCOEF$ are horizontally homogeneous and if
    refraction of light and Earth's curvature are neglected. These restrictions
    are further discussed in section \in[ch:elevation_scanning].

\stopsection


\startsection[title={Atmospheric Extinction and Absorption Models},reference=ch:extinction]

    The calculation of brightness temperatures using \ineq{rte_solution}
    requires the determination of atmospheric absorption/emission coefficients
    $\ABSCOEF$. The main properties of atmospheric absorption in the microwave
    region from 20 to 200 GHz are given here based on a review by
    \cite[Westwater2004] and some additions from other authors.

    \placefigure[top][fig:absorption]
        {Absorption coefficients (black) and contributions of individual terms
        (other) according to the models of \cite[Liebe1993] (gaseous) and
        \cite[Turner2016] (cloud) for an atmospheric layer of 850 hPa pressure,
        0°C temperature and saturation, containing 0.1 g/kg of liquid water.
        The arrows mark frequencies of the channels of the HATPRO radiometer
        which is used for subsequent retrievals.
        }
        {\externalfigure[absorption][width=\textwidth]}

    The three main constituents in the atmosphere contributing to extinction
    and emission of microwave radiation are oxygen, water vapor and liquid
    water. A typical absorption spectrum of an
    atmospheric layer is shown in Figure \in[fig:absorption]. Water vapor has
    a relatively weak absorption band around 22.235 GHz (K band) and oxygen
    a stronger band around 60 GHz (V band). Outside of the shown frequency
    range exist an oxygen absorption line at 118.75 GHz and a stronger water
    vapor line at 183.31 GHz. The wings of these lines and so called continuum
    absorption associated with pressure broadening processes and other
    atmospheric species also contribute to the overall absorption
    \cite[authoryears][Liebe1993,Hewison2006]. Finally, in the presence of
    clouds there is absorption by water droplets. Due to the
    size-wavelength-ratio of droplets in non-precipitating clouds and microwave
    radiation, scattering can be neglected and the Rayleigh-approximation is
    apropriate for the description of absorption of radiation by liquid water.
    Absorption and scattering by snow and ice in the atmosphere has negligible
    effect on radiation in the K and V bands \cite[authoryears][Sanchez2013].
    During rain events however scattering of microwave radiation by
    precipitation cannot be neglected and rainwater accumulation on the
    radiometer's radome must be accounted for as well. Since these effects are
    much more difficult to quantify than absorption they are usually not
    included in radiative transfer models and the resulting errors are accepted
    or retrievals are not performed during precipitation at all.

    The absorption spectrum in Figure \in[fig:absorption] was calculated by
    a model from \cite[Liebe1993] for gaseous absorption and a model from
    \cite[Turner2016] for absorption by cloud liquid water. There are other
    of such models in use, common are a model by \cite[Rosenkranz1998], MonoRTM
    \cite[authoryears][Clough2005] and ARTS \cite[authoryears][Buehler2005]
    (the latter two are not just absorption models but entire radiative
    transfer simulators).  These models usually account for the most important
    absorption lines explicitly, correct for missing ones and parameterize the
    continuum term by an empirical function. The \cite[Liebe1993] model for
    example describes oxygen line absorption in the microwave region by 44
    lines and water vapor by 34 lines. Cloud absorption is most often
    parameterized by a double-Debey model
    \cite[alternative=authoryears,left={(e.g. }][Turner2016]. Instead of only
    describing atmospheric absorption, absorption models are formulated in
    terms of the more general refractivity $N$ which is the complex refractive
    index reduced by one. With knowledge of $N$, absorption can be calculated
    by

    \startformula
        \ABSCOEF = \frac{4 \pi \nu}{c}~ {\rm Im}(N)
    \stopformula

    where $c$ is the speed of light and $\nu$ is frequency.

\stopsection


\startsection[title={State Vector Variables},reference={ch:statevector}]

    The emission and absorption properties of the atmosphere depend on its
    state which has to be represented in some form for the retrieval. Directly
    using the quantites that are to be retrieved (temperature, humidity, liquid
    water content) as state vector variables works well for regression
    retrievals. Any method that makes the assumption of Gaussian distributed
    state vector variables however possibly requires variable transformations
    to achieve value distribution that approximately follow a Gaussian curve.
    For optimal estimation retrievals it is common to combine water vapor and
    liquid water into a single quantity and then retrieve the logarithm of this
    quantity
    \cite[alternative=authoryears,left={(e.g. }][Lohnert2004,Hewison2006,Cimini2011].
    Figure \in[fig:gauss_verification] shows the benefits of using the
    logarithm of humidity with an example from a single height level of
    a climatology based on radiosonde measurements from Innsbruck airport.
    The distribution of temperature is very close to Gaussian, but
    humidity\footnote{In this case specific humidity but the same applies to
    water vapor density (absolute humidity) and relative humidity} is
    a non-negative variable with a skewed distribution. Applying the logarithm
    reduces the skewness of the distribution, brings it closer to a Gaussian
    shape and has the additional positive effect of enforcing the positivity
    of humidity values during retrieval.

    \placefigure[top][fig:gauss_verification]
        {Histograms of temperature (left), specific humidity (center) and
        the natural logarithm of specific humidity (right) for the 1366 m
        level of the radiosonde training data set (section \in[ch:rasoclim]).
        The black curves are Gaussian distributions fitted to respective data.
        }
        {\externalfigure[gauss_verification][width=\textwidth]}

    Reducing water vapor and liquid water to a combined total water content
    variable is practical in multiple ways. In terms of the distribution
    functions the asymmetry found for liquid water content is even worse than
    for water vapor since many atmospheric profiles are cloud-free resulting
    in a second mode at value zero. Because the logarithm is not defined at
    zero this particular variable transformation to reduce skew is not
    applicable. But since there always is some water vapor in the atmosphere
    this mathematical problem can be avoided by adding both components of water
    content and then applying the logarithm transformation. Another advantage
    of the combined variable is the reduction of state vector dimensionality
    resulting in computationally less expensive optimal estimation retrievals.

    The combination of all atmospheric water into a single variable is only
    sensible if there is a way to separate the components again for the
    calculation of the absorption coefficients. The effect of ice on microwave
    radiative transfer is negligible (see section \in[ch:extinction]) therefore
    it is sufficient to only include water vapor and liquid in the total water
    content. The governing quantity of the partitioning scheme for these
    components is the saturation vapor pressure of air over water as it
    determines when the capabilities of the gaseous water content are exceeded
    and cloud droplets start to form\footnote{The possibility of supersaturated
    air complicates this relation as other factors such as cloud condensation
    nuclei become important. This is neglected here.}. Becuase saturation vapor
    pressure is a function of temperature and temperature is included in the
    retrieved state vector the water separation threshold can be calculated
    during the retrieval. The chosen partition function should be
    differentiable to stabilize the retrieval. \cite[Hewison2006] approached
    this problem by introducing a transition region to a function of
    \cite[Deblonde2003], which gradually introduced liquid water content
    starting at a given relative humidity threshold. The chosen form of
    partition function here is

    \startformula
        \DERIV{\QLIQ}{\RHL} = \QSAT \startcases
            \NC 0 \MC s \lt 0.95 \NR
            \NC \cos \left( \frac{\RHL - 1.05}{0.1} \frac{\pi}{2} \right)^2
                \MC 0.95 \le s \le 1.05 \NR
            \NC 1 \MC 1.05 \lt s \NR
        \stopcases
    \stopformula

    where

    \startformula
        \RHL = \frac{\QTOT}{\QSAT} = \frac{\QVAP + \QLIQ}{\QSAT}
    \stopformula

    is the fraction of total water content to water content at saturation,
    a generalization of relative humidity that includes liquid water.
    Integration yields the desired expressions for liquid water content and
    specific humidity

    \placeformula
    \startformula
    \startalign[align={center,right}]
        \NC
        \QLIQ(\RHL) = \startcases[align={left,left,right}]
            \NC 0 \MC s \lt 0.95 \NR
            \NC \frac{\QSAT}{2} \left( \RHL - 0.95 - \frac{0.1}{\pi}
                \cos \left( \frac{\pi \RHL}{0.1} \right) \right)
                \MC 0.95 \le s \le 1.05 \NR
            \NC \QLIQ(1.05) + \QSAT (\RHL - 1.05) \MC 1.05 \lt s \NR
        \stopcases \NC \NR[eq:qliq]
    \stopalign
    \stopformula

    and

    \placeformula[eq:qvap]
    \startformula
        \QVAP(\RHL) = \QTOT(\RHL) - \QLIQ(\RHL) \EQSTOP
    \stopformula

    It uses 95 \% relative humidity as the threshold for the introduction of
    cloud liquid water. Saturation is reached at 105 \% relative humidity and
    any remaining water contributes to the liquid component.

    It is important to note that because temperature is critical for
    determination of the saturation pressure of water vapor the introduction of
    such a partion of water content results in a strong dependency of the
    humidity retrieval on the temperature retrieval
    \cite[authoryears][Bleisch2012]. It is therefore impossible to retrieve
    humidity independently of temperature and the direct connection between
    relative humidity and cloud liquid water restricts the possible states
    of the atmosphere. The \cite[Karstens1994] cloud model used in this thesis
    to extract information on liquid water content from profiles obtained by
    a radiosonde ascent assumes cloud liquid water exists wherever relative
    humidity exceeds 95 \% but then calculates the liquid water content
    independently of the exact value of relative humidity. The output of
    the numerical weather prediction model COSMO-7 whose forecasts are used
    later to provide prior distributions for optimal estimation retrievals
    fixes relative humidity at 100 \% in all clouds. In both cases
    given combinations of humidity and cloud water may not be representable by
    the partition \ineq{qliq} and \ineq{qvap}. In practice the two components
    are nevertheless added and any changes after separation are accepted. The
    uncertainty in the determination of cloud liquid water from models like the
    one by \cite[Karstens1994] is hard to assess \cite[authoryears][Ebell2010]
    and potentially large, so the convenience of the partition function
    outweighs the additional inaccuracies resulting from the restriction of
    humidity state space by the variable combination.

    For optimal estimation retrievals there is additionally the need to update
    pressure information after each iteration since pressure is important for
    the calculation of absorption coefficients and depends on temperature and
    humidity but is not part of the explicitly retrieved atmospheric state
    variables. Pressure is calculated after each iteration by numerical
    integration of the hydrostatic equation

    \startformula
        p(z) = p(z_0) + \exp \left( - g \int_{z_0}^{z} \frac{1}{R(h) \, T(h)} \, \diff h \right)
    \stopformula

    where $z_0$ is the height of the radiometer and $p(z_0)$ is provided by
    a pressure sensor at that height (which is built into the radiometer)
    and $R$ is the specific gas constant of air which is a function of height
    due to its dependence on water vapor content. The errors of this
    approximation have been assessed by a comparison of hydrostatic
    approximations with the pressure information from radiosonde ascents and
    were found to be negligible.

\stopsection


\startsection[title={Retrieval Information Content and Weighting Functions}]

    Two important properties associated with atmospheric absorption govern the
    frequencies of channels which a microwave radiometer observes in order
    to retrieve vertical profiles of thermodynamic variables. Temperature
    information can be obtained from emission and absorption of oxygen which
    are directly proportional to local temperature and independent of mixing
    ratio because oxygen is a well mixed gas throughout the atmosphere.
    Water vapor amounts on the other hand vary strongly with height but with
    if temperature is known, the emission of water vapor is proportional to its
    partial density \cite[Westwater2004]. To obtain vertical profiles, channels
    along the sides of absorption bands like the K and V band are observed.
    Each channel allows to look differently far into the atmosphere due the
    different opacity of the atmosphere at each frequency
    \cite[authoryears][Churnside1994]. Unfortunately measurements at different
    frequencies along the same absorption band are not independent of one
    another therefore the information content of observations is not
    proportional to the number of radiometer channels
    \cite[authoryears][Hewison2006]. To increase the vertical resolution
    without the adding new channels the radiometer antenna can be pointed
    away from zenith, resulting in longer attenuation
    paths which provide more information on the lower layers of the atmosphere.
    This procedure is called elevation scanning and is further discussed in 
    section \in[ch:elevation_scanning].

    A common visualization of the information content of a radiometer
    observation are the so-called weighting functions each corresponding to
    one brightness temperature observation. Weighting functions are obtained
    from radiative transfer models by taking the derivative of the measured
    brightness temperature with respect to the atmospheric state vector. In the
    case of a vertically discretized atmosphere the weighting functions are the
    rows of the Jacobian of the forward model, containing the derivatives with
    respect to each state vector element \cite[Rodgers2000]. The weighting
    functions therefore express the linearized response of the radiative
    transfer to a change in the atmospheric state. Because weighting functions
    have the same dimension as the atmospheric state vector they can
    conveniently be visualized on the retrieval grid. Their magnitude can be
    interpreted as the sensitivity of the measurement to the considered state
    vector component at the given height. Hence weighting functions can be used
    to assess from which heights in the atmosphere a radiometer channel obtains
    its information \cite[authoryears][Martinet2015]. The optimal estimation
    framework directly incorporates the forward model Jacobian in the inversion
    process. The individual weighting functions act similarly to basis
    functions which determine where and at which scales the profile can be
    modified during the retrieval. The functions' smoothness is therefore
    a limiting factor on the vertical resolution of features of the atmospheric
    state which can be recovered by the retrieval from a radiometer
    observation. This highlights the importance of the first guess in the
    optimal estimation framework. Its small scale features may be fixed during
    the retrieval if the weighting functions smooth.

    Derivation of the Jacobian from any numerical radiative transfer model is
    possible by finite differencing. This brute force approach is very flexible
    but computationally expensive and can result in inaccurate linearizations
    if perturbations are selected without care. Model developers have therefore
    started to implement routines dedicated to the accurate calculation of
    Jacobians based on other differentiation techniques. ARTS
    \cite[authoryears][Buehler2005] and the recently developed RTTOV-gb
    \cite[authoryears][DeAngelis2016] provide such procedures and the numerical
    model implemented at the end of this chapter is also able to calculate
    exact Jacobians without finite differencing.

    \placefigure[top][fig:jacobian_frequency]
        {Normalized weighting functions for the V band (left) and K band
        (right) channels of the HATPRO instrument (see section \in[ch:hatpro])
        simulated by the radiative transfer model introduced in section
        \in[ch:mwrtm] for an upwards looking view. The ground is at an altitude
        of 612 m. V band weighting functions are given with respect to
        temperature, K band weighting functions with respect to the natural
        logarithm of humidity. Lighter colors correspond to channels of higher
        frequency. The assumed atmospheric profile is the U.S. standard
        atmosphere with relative humidity decreasing from 70 \% at the surface
        to 10 \% at 11 km above which it is constant.  }
        {\externalfigure[jacobian_frequency][width=\textwidth]}

    Figure \in[fig:jacobian_frequency] shows weighting functions for HATPRO
    channels in the V band with respect to temperature and in the K band with
    respect to humidity. Temperature weighting functions are more diverse in
    their shape than those of humidity. This indicates that the vertical
    resolution of temperature retrievals is generally better than that of
    humidity retrievals \cite[authoryears][Cimini2011]. Information on humidity
    mainly comes from the lower part of the troposphere. This is largely due
    to the much higher concentrations of water vapor at these heights as the
    saturation vapor pressure decreases exponentially with decreasing
    temperature. The V band channels of lower frequency show sensitivity to
    mid-tropospheric temperatures while information from channels of higher
    frequency is restricted to the lower troposphere.

\stopsection


\startsection[title={Retrieval Grid},reference={ch:retrievalgrid}]
  
    Section \in[ch:statevector] discussed issues regarding the choice of
    variables used to represent the atmospheric state and used the concept of
    a state vector. This vector is the result of a vertical discretization
    of the atmospheric profile. This discretization is based on the retrieval
    grid, a set of altitudes at which the state variables are determined.

    At first it seems sensible to choose a retrieval grid with a resolution as
    high as possible in order to obtain the most accurate retrievals. Most
    studies however use only around 40 to 60 vertical levels for the
    discretization of the entire tropopause
    \cite[alternative=authoryears,left={(e.g. }][Solheim1998,Hewison2006,Martinet2015].
    Because the ability of a retrieval to resolve small scale features of the
    atmospheric state is limited by the smoothness of the weighting functions.
    Considering the shape of the weighting functions in Figure
    \in[fig:jacobian_frequency], the retrieval grid should have a higher
    resolution in the lower troposphere but at a certain point increases in
    resolution will not be reflected by the retrieval performance. This was
    confirmed by \cite[Meyer2016] who observed no improvement of accuracy
    in an experiment with a retrieval grid of enhanced resolution. Practically,
    the vertical discretization should be chosen such that it is not much finer
    than the actual vertical resolution of the radiometer since less levels
    result in higher computational performance of the retrieval technique. The
    cost of calculating the Jacobians in particular scales with the number of
    levels of the retrieval grid and is usually the most expensive process in
    an optimal estimation retrieval.

    Special consideration must be given to the altitude of the highest level
    of the retrieval grid.  Radiometer channels measuring in the more
    transparent regions of the microwave spectrum can have significant
    contributions by emission of the tropopause and lower stratosphere. If such
    parts of the profile are not represented by the atmospheric state vector
    they should be corrected for in radiative transfer calculations otherwise
    simulated brightness temperatures can be significantly biased. This issue
    will be discussed further in sections \in[ch:missing_profile] and
    \in[ch:model_comparison].

\stopsection


\startsection[title={Elevation Scanning},reference={ch:elevation_scanning}]

    Many radiometers are able to rotate their antenna in order to measure
    brightness temperatures at off-zenith angles, a procedure called elevation
    scanning. Changing the angle of measurement changes the associated
    weighting functions and therefore the heights from which the channels get
    their information on the atmospheric state. This can increase the overall
    information content of an observation and result in a higher vertical
    resolution of the retrieval in the boundary layer
    \cite[authoryears][Westwater2004] while the impact on retrieval performance
    at higher levels is only small \cite[authoryears][Cimini2006]. Figure
    \in[fig:jacobian_angle] shows weighting functions for two frequencies in
    the V band at different angles. When the zenith angle is larger, i.e. when
    the radiometer view is more horizontal, the sensitivity of the channels
    increases near the surface while decreasing higher up.

    \cite[Xu2014] remarked that due to the U-shape of the radomes used by many
    microwave radiometers measurements from elevation scans are less affected
    by rainwater accumulation on the instrument than those at zenith. This
    effect was previously observed by \cite[Cimini2011] who noticed that
    off-zenith signals during precipitation were less noisy than zenith
    observations.

    \placefigure[top][fig:jacobian_angle]
        {Normalized weighting functions for 54.94 GHz (left) and 58.00 GHz
        (right) for different zenith angles simulated by the radiative transfer
        model introduced in section \in[ch:mwrtm]. The 0° weighting function
        (upward looking) is the blue line, angles 60°, 65°, 70°, 75°, 80°, 85°
        are the gray lines, with lighter colors refering to larger zenith
        angles. The atmospheric profile and radiometer altitude are the same as
        in Figure \in[fig:jacobian_frequency].
        }
        {\externalfigure[jacobian_angle][width=\textwidth]}

    Elevation scanning puts additional demands on the instrument, retrieval and
    radiative transfer model. In order to accurately observe brightness
    temperatures at large zenith angles, radiometers be very sensitive.
    High sensitivity is usually achieved by wide bandwidths of the channels
    used at off-zenith angles \cite[authoryears][Cadeddu2002,Crewell2007].
    Because the execution of an elevation scan requires additional time,
    a trade-off between temporal resolution and averaging time of a measurement
    is necessary \cite[authoryears][Cadeddu2002]. Measurements in transparent
    channels are very sensitive to the misalignment of angles so the antenna
    adjustment must be precise \cite[authoryears][Hewison2006]. In complex
    terrain the radiometer view might intersect with a mountain and surface
    emission by the slope must be considered. Violations of the assumption of
    horizontal homogeneity are a major problem of elevation scanning in general
    as the likelihood that the radiometer view includes different air masses
    rises the more horizontal it is \cite[authoryears][Cimini2006]. This is
    particularly an issue in situations of inhomogeneous cloud cover
    \cite[authoryears][Guldner2013]. Effects of inhomogeneity can be reduced by
    scanning in multiple directions and averaging the measurements from same
    zenith angles. This increases the time to make an observation further.
    \cite[Cimini2006] noticed a strong impact of horizontal inhomogeneity on
    neural network retrievals of temperature and humidity, likely due to the
    non-linearity of the method amplifying fluctuations in the brightness
    temperatures. \cite[Chan2010] used information from off-zenith angles only
    for the lowest levels of the atmosphere when retrieving humidity profiles
    by linear regression.

    Many of the mentioned complications can be avoided by excluding transparent
    channels in retrievals using elevation scan data.
    \cite[Crewell2007,Massaro2015] used only the most opaque channels in the
    V band at off-zenith angles for this reason. Aside from these
    considerations, subsequent retrievals in this work use only the four most
    opaque channels in the V band due to a deficiency of the radiative transfer
    model. Off-zenith light paths in the atmosphere are affected by refraction
    caused by density changes of air with height and Earth's curvature must be
    taken into account as well \cite[authoryears][Hewison2006]. The radiative
    transfer model introduced in section \in[ch:mwrtm] currently does not
    adjust light paths for refraction which causes large errors in simulated
    brightness temperatures of transparent channels at off-zenith angles.
    Paths lengths of light in the opaque region of the V band are very short
    \cite[alternative=authoryears,left={(300 - 2000 m, }][Crewell2007] and
    refraction of light is negligible.

\stopsection


\startsection[title={Characterization of Errors},reference={ch:rtm_errors}]

    Error sources:
    FAP
    finite bandwith of radiometer channels while model usually assume
    monochromatic frequency
    profile discretization/representation
    choice of profile extension to account for stratospheric emission
    uncertainties in angular position when elevation scanning
    radiometric noise
    calibration issues
    radio frequency interference (other sources of microwave radiation than
    atmosphere, especially relevant in highly populated areas), these are
    hard to assess and likely very variable, therefore usually not accounted
    for
    representativeness (temporal averaging of measurement, horizontal
    inhomogeneity)

    bias vs. variation (assessed via root mean square error)

    problems of verification and determination: no perfect RTM, complications
    with reference instruments, mainly radiosondes as these are affected by
    errors as well (drift, representativity).

    Instrument random noise and forward model errors important for retrieval
    \cite[Caddedu2002]. Instrument noise of radiometers often assumed to be
    0.5 K and Gaussian \cite[Chu1994,Frate1998,Lohnert2004]

    Building part of the forward model covariance matrix.

    Compare exact Jacobian with brute force Jacobian.

    FAP errors: determine from climatology by comparison of exact
    calculations with FAP predictions.

    Discretization errors: deterine from climatology by comparing
    interpolated profiles with full resolution profiles.

    Bias correction important
    \cite[Turner2007,Turner2013,Lohnert2012,Guldner2013,Martinet2015] but
    \cite[Cimini2011] found this is not that big of an issue....

    Calibration of radiometer can cause jumps in retrieval performance due to
    bias changing \cite[Lohnert2012,Guldner2013].

    Uncertainty of RT is major limit on retrieval especially water
    vapor \cite[Cimini2004] and uncertainties in models can reach up to 4K in
    terms of brightness temperatures.

\stopsection


\startsection[title={Implementation of a Radiative Transfer Model Prototype},reference=ch:mwrtm]

    A numerical microwave radiative transfer model has been implemented as
    part of this thesis. It is based on the absorption models by
    \cite[Liebe1993] for gaseous absorption and \cite[Turner2016] for cloud
    liquid water absorption. The prototype's distinguishing features are its
    accessible implementation in a high-level, dynamic programming language
    (Python 3) and the calculation of temperature and humidity Jacobians by
    forward-mode automatic differentiation. Given an atmospheric profile of
    pressure, temperature and humidity, it first calculates absorption
    coefficients and then evaluates equation \ineq{rte_solution} numerically.
    The model will subsequently be referred to as MWRTM (MicroWave Radiative
    Transfer Model).

\startsubsection[title={Motivation}]

    It is not obvious why another implementation of a numerical radiative
    transfer model is attempted when established and freely available software
    such as MonoRTM \cite[authoryears][Clough2005] or ARTS
    \cite[authoryears][Buehler2005] exists. The development of a new model
    prototype is motivated by three aspects:

    Current RTMs either resort to expensive finite difference approaches when
    calculating Jacobians or use hand-crafted routines mixing analytically
    derived expressions and finite difference approximations where analytical
    expressions are unavailable. As examples, MonoRTM currently has no
    implementation of derivatives making finite differencing necessary while
    ARTS has specially written modules for calculating Jacobians. The rising
    interest in machine learning in recent years has lead to the widespread use
    of automatic differentiation, a method with which any numerical program can
    be differentiated to machine precision without the need to explicitly write
    code for the derivatives. This technique has proven its usefulness in
    machine learning models and is used in the radiative transfer model for the
    exact calculation of weighting functions.  Current radiative transfer
    models are written in statically typed, compiled languages such as Fortran
    or C++. Data analysis however is usually done in a high-level language such
    as Python. A question of interest is if it is computationally feasible to
    implement a RTM completely in a dynamic language. This would make the model
    code more accessible to the user and simplify the setup procedure. Finally
    there is a personal motivation of the author associated with the decision
    to implement a radiative transfer model. It is assumed that detailed
    knowledge of all implementation aspects and calculations will help with the
    overall understanding of retrievals and their results.

    It is not expected that the implementation attempt directly results in
    a perfect model. The expectation is rather to build a flexible, easy to
    understand, reasonably fast and adequately accurate prototype. The next
    sections discuss important aspects of development and evaluate the
    overall model performance.

\stopsubsection

\startsubsection[title=Numerical Considerations]

    In order to obtain numerically stable results the radiative transfer model
    has to operate internally with a much finer discretization than that of
    typical grids on which retrievals take place. Due to the generally greater
    changes of absorption and temperature with height in the lower troposphere,
    the internal model grid can be coarser at greater altitudes without losing
    accuracy.

    User has full control over internal model discretization, choice for
    subsequent retrievals is a logarithmically spaced grid.

    Input has to be interpolated to internal grid. First try was calculating
    absorption coefficients first and linearly interpolating those but this
    resulted in large errors. It was then decided to first interpolate
    pressure, temperature and humidity to the model grid and to calculate
    the absorption coefficients on the fine grid.

    Quadrature scheme: second order trapezoidal rule.
    
    Error covariance.

\stopsubsection

\startsubsection[title=Automatic Differentiation,reference=ch:autodiff]

    Interest in efficient and accurate computation of gradients has risen
    immensely in recent years due to its need during the training of many
    machine learning algorithms. The concept has been known at least since the
    1960s and been rediscovered multiple times since then
    \cite[authoryears][Griewank2012]. The need to efficiently calculate
    Jacobians of numerical models (often called adjoint modelling) is also
    a problem often encountered in atmospheric sciences. Data assimilation for
    numerical weather prediction models was one of the first applications of
    automatic differentiation \cite[authoryears][Griewank2012]. Despite that,
    the technique was mentioned only in one short paragraph modelling by
    \cite[Errico1997] in a review of adjoint models and even a modern radiative
    transfer model such as ARTS still has hand-written code for the calculation
    of Jacobians.

    At its core, automatic differentiation is nothing more than the application
    of the chain rule of differentiation to the computational graph of
    a program. It is not the same as symbolic differentiation though, which
    often requires significant effort and has the problem of combinatorial
    blowup of expressions. Instead, implementations of automatic
    differentiation are able to compute derivatives simultaneously with the
    forward code or in a second pass and a decent implementation has to visit
    each edge of the computational graph only once resulting in equal
    complexities of forward and Jacobian calculations. A short and good
    (although not peer-reviewed) explanation of the idea behind automatic
    differentiation is found at
    \hyphenatedurl{http://colah.github.io/posts/2015-08-Backprop}
    \footnote{last accessed 2016-07-25}.

    There are two ways of performing automatic differentiation, differentiated
    by the direction of traversal of the computational graph. The forward-mode
    moves from the input nodes towards the output, computing the derivatives
    with respect to each input variable along the way. Implementations of
    forward-mode automatic differentation are easy in programming languages
    supporting dispatch on types. Aside from replacing numeric variables with
    instances of a custom type with additional fields for the derivatives and
    overloading the usd operators and functions, only minor or no changes to
    the program at all are necessary. The reverse-mode moves from the output
    leafs toward the input nodes, computing the derivative of each output with
    respect to each node of the computational graph. Because the reverse-mode
    requires a second pass thorough the graph implementations need to keep
    track of intermediate results calculated in the first pass. Both techniques
    yield the same results but can be very different in terms of computational
    cost. Generally speaking, reverse-mode is suitable for problems with many
    inputs and only few outputs while the forward mode is more efficient for
    problems with more outputs than inputs.

    A radiative transfer model, with many atmospheric layers as input and only
    a few brightness temperatures as output is better suited for reverse-mode
    differentiation. The prototype here nevertheless uses the much easier to
    implement forward-mode. Transformation of the forward model code required
    a custom type, the overloading of a few operators and functions and minimal
    changes to the existing code. The only performance optimization made is
    a special consideration of derivatives associated with the calculation of
    the absorption coefficients on the internal model grid.

\stopsubsection

\startsubsection[title=Fast Absorption Prediction]

    The radiative transfer model spends the majority of its time calculating
    absorption coefficients. Although the models of \cite[Liebe1993] and
    \cite[Turner2016] are already fairly simple, it is convenient to replace
    them by a fast absorption predictor (FAP) in order to reduce the overall
    model computation time. FAPs are regression models of the absorption
    coefficient as a funcion of the state vector, trained on data representing
    typical atmospheric conditions. fast absorption predictors have previously
    been used by \cite[Lohnert2004], \cite[Hewison2006] and others.

    Following the choice of state vector variables from section
    \in[ch:statevector], the predictors are chosen to be pressure, temperature
    and the natural logarithm of total specific water content. A first
    implementation modelled both gaseous and cloud absorption together using a
    polynomial of the state vector variables. Even for polynomials of degree
    five including all interaction terms large errors were found in the
    approximated absorption coefficients. Investigation of this issue showed
    that the polynomial was unable to adequately represent the strong
    non-linearity of absorption near the cloud threshold. Polynomials of higher
    order are strongly affected by combinatorial blowup of interaction terms
    and numerical stability issues, therefore a more refined approach is taken.

    Instead of fitting a polynomial to the combined gaseous and cloud
    absorption the components are approximated separately after the total
    water content of the state vector has been separated into vapor and liquid
    components according to \ineq{qliq} and \ineq{qvap}. Specific cloud
    absorption following \cite[Turner2016] is only a function of temperature
    modelled by the FAP as a 5th order polynomial of temperature. Gaseous
    absorption depends on $p$, $T$ and $\QVAP$. \cite[Lohnert2004] and
    \cite[Hewison2006] used third order polynomials without interaction terms
    as FAPs and claimed good accuracy. Here it was found to be necessary to
    approximate absorption coefficients by third order polynomials including
    all interaction terms for V band frequencies and third order polynomials
    including interactions terms up to second order for K band frequencies. The
    higher order polynomial for K band frequencies was found to be necessary to
    avoid negative absorption coefficients in the more opaque channels. Before
    it was also tested to model the logarithm of the absorption coefficient
    with an FAP to enforce positivity. Due to a negative impact on accuray,
    likely due to the additional non-linearity introduced by the
    transformation, this was not pursued further.

    Errors introduced by the FAPs are quantified in an error covariance matrix
    and included in the total forward model error. The matrix is determined
    from differences of simulations with and without fast absorption
    predictors.

\stopsubsection

\startsubsection[title={Correcting for Missing Parts of the Profile},reference=ch:missing_profile]

    Mentioned before in retrieval grid section. Multiple things to consider:
    information content of radiometer mostly limited to lower atmosphere (see
    weighting functions). Absorption becomes increasingy non-linear at pressure
    lower than 150 hPa, therefore FAPs perform badly or have to use
    higher-order polynomials which are more expensive, defeating the purpose of
    the FAP.

    Choice here: 12 km above ground is end of profile. Transparent channels
    have contributions from above. K band is irrelevant since water vapor
    content at these height is negligible. V band significantly affected by
    emission from tropopause and lower stratosphere. If not accounted for,
    modelled BTs are biased. MWRTM provides the possiblility to account for
    the missing emission from above the profile end by adjusting the cosmic
    background temperature to a value that includes the emission and opacity
    of the upper atmosphere. The assumed values of temperature and pressure
    come from the US standard atmosphere. Humidity is set to zero.

\stopsubsection

\startsubsection[title={Model Comparison},reference={ch:model_comparison}]

    The model should be compared to established radiative transfer models to
    ensure its correctness. A model comparison also allows a quantification of
    spectroscopy errors which are an important part of the forward model error
    covariance (section \in[]) \cite[authoryears][Hewison2006]. Brightness
    temperature simulations from two other models are available for the
    evaluation: MonoRTM version 5.2 \cite[authoryears][Clough2005] and a model
    using gas absorption according to \cite[Rosenkranz1998] previously used by
    \cite[Massaro2015]. The test data set is made up of 1568 radiosonde
    profiles from Innsbruck. Because the data from \cite[Massaro2015] use
    a slightly different cloud liquid water parameterization than this thesis
    does (see section \in[ch:???]) and only spectroscopic errors should be
    assessed here all profiles of the test data set are clear sky cases.

    \placefigure[top][fig:model_comparison]
        {Mean difference (saturated) and covariance (pale) of brightness
        temperatures calculated by the model implemented in section
        \in[ch:mwrtm] (MWRTM), MonoRTM 5.2 and a model based on absorption
        as described by \cite[Rosenkranz1998] for over 1000 radiosonde ascents
        and zenith view. The covariances are sampled from the full covariance
        matrices.
        }
        {\externalfigure[model_comparison][width=\textwidth]}

    Figure \in[fig:model_comparison] shows the results of the model comparison
    in terms of mean and standard deviation of model differences. As noticed
    before by \cite[Westwater2004] and \cite[Cimini2004], the absorption models
    agree best in opaque spectral regions. Biases change sign between V and
    K band and their magnitude in the more transparent V band channels is
    generally higher than in the K band channels. Overall, MWRTM agrees better
    with MonoRTM than with Rosenkranz and MonoRTM and Rosenkranz agree better
    with each other than with MWRTM.

    Magnitudes of the bias values can be compared to a study by
    \cite[Cimini2004] who evaluated four absorption models including all of the
    ones used here (although they used an older version of MonoRTM).  Overall
    bias values on the order of 1 - 2 K are found in their comparison as well.
    In the K band their implementation of a RTM using \cite[Rosenkranz1998]
    absorption also showed better agreement with MonoRTM than with the
    \cite[Liebe1993] model. In the V band however agreement between absorption
    by \cite[Rosenkranz1998] and \cite[Liebe1993] is better than seen in Figure
    \in[fig:model_comparison] with a significantly smaller bias in the more
    transparent channels.

    TODO:
    Bias in transparent oxygen channels very sensitive to choice of highest
    level of retrieval grid and method of accounting for missing parts of
    atmosphere. This kind of error is mostly a bias which has no influcence
    when evaluating synthetic retrieval performance and will be corrected for
    when using actual radiometer observations.

    \cite[Liebe1993] absorption verified by comparison to routines from
    \hyphenatedurl{http://reef.atmos.colostate.edu/~odell/at622/assignments/project1/codes/},
    no difference if exact absorption of FAP is used, no difference if internal
    model resolution is changed.
    Bias does not appear to scale with transparency of channel and changes sign
    between K and V band.
    Since comparison results are from zenith refraction cannot be cause of
    differences either.

    Standard deviation higher for more transparent channels.

    Also compare with \cite[Lohnert2004], figure 2.

    Elevation scanning. TODO:
    This is also confirmed by a comparison of simulations with results from the
    reference model from section \in[ch:model_comparison] which shows
    differences considerably lower than the noise level of the radiometer.


    The spectroscopic error covariance matrix is assembled from the comparison
    with the model used by \cite[Massaro2015].

\stopsubsection

    
\startsubsection[title={Achievements and Shortcomings}]

    The implementation of MWRTM's automatic differentiation is rather crude and
    not extremely efficient. Nevertheless it showcases that the method is
    applicable to radiative transfer modelling and even though barely any
    performance optimizations were done, calculation of Jacobians is
    significantly faster than by finite differences. Future developments with
    automatic differentiation should apply one of the many available code
    generators, which make it easy to use backward-mode differentiation and
    exploit compiler optimizations for high efficiency.

    This discrepancy might require more investigation if
    the MWRTM is developed further but for the rest of this thesis it is still
    the model used for all retrievals. This is justified by the following
    arguments: Since there is no bias between the MWRTM and other models in
    the more opaque channels of the V band and the standard deviation between
    models is less than 1 K even for the transparent channels, it is unlikely
    that the source of this discrepancy is a fundamental error in the
    implementation of the \cite[Liebe1993] absorption model. A more likely
    source are different discretization, interpolation and numerical quadrature
    schemes between the model implementations or differences in the test data
    sets (the test data set of \cite[Cimini2004] contained only 34 profiles).
    In practice, when assessing the performance of synthetic retrievals,
    a model bias is irrelevant as long as the entire retrieval is affected by
    the same bias. In applications involving real radiometer measurements model
    biases are determined by a comparison between matching observations and
    model calculations and subsequently accounted for. The observed discrepancy
    will therefore have no effect on the retrieval performance investigated
    here.

\stopsubsection

\stopsection

