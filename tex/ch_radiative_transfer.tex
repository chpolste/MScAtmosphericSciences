\startsection[title=The Radiative Transfer Equation]

    General form, absorption only, Rayleigh-Jeans → Brightness Temperatures.
    Discretization.

\stopsection


\startsection[title=Representation of the Atmospheric State]

    Discretization in the vertical, selecting state vector variables:
    compromise between dimensionality and information content (including
    forward model needs).

    \startsubsection[title=The Choice of State Vector Variables]

        Pressure as fixed variable (recalculated by hydrostatic equilibrium
        after each iteration, check if this is good approximation when compared
        to radiosonde measurements). Temperature and humidity as variable
        state quantities. To reduce dimensionality: use knowledge of qsat to
        combine water vapor and liquid water content into a single variable
        qtot. No representation of ice due to negligible absorption in
        microwave region (see below INSERT REF). The individual components are
        then recovered by a partition function

        \startformula
            \DERIV{\QLIQ}{\RHL} = \frac{1}{\QSAT} \startcases
                \NC 0 \MC s \lt 0.95 \NR
                \NC \cos \left( \frac{\RHL - 1}{0.05} \frac{\pi}{2} \right)^2
                    \MC 0.95 \le s \le 1 \EQCOMMA\NR
                \NC 1 \MC 1 \le s \NR
            \stopcases
        \stopformula

        where

        \startformula
            \RHL = \frac{\QTOT}{\QSAT} = \frac{\QVAP + \QLIQ}{\QSAT} \EQSTOP
        \stopformula

        Different values than hewison or deblonde, to simplify calculation
        of qtot from radiosondes and model data: everything over 100 \% RH
        is assumed to be liquid water (COSMO7 fixes RH at 100 \% in clouds).
        Integrate, get values for qliq and qvap.

        \startformula
            \QLIQ(\RHL) = \QSAT \startcases
                \NC 0 \MC s \lt 0.95 \NR
                \NC \frac{1}{2} \left( \RHL - 0.95 + \frac{0.05}{\pi}
                    \sin \left( \frac{\pi \RHL}{0.05} \right) \right)
                    \MC 0.95 \le s \le 1 \NR
                \NC \QLIQ(1) + \RHL - 1 \MC 1 \le s \NR
            \stopcases
        \stopformula

        and from REF to s = qtot/qsat

        \startformula
            \QVAP(\RHL) = \QSAT \RHL - \QLIQ(\RHL) \EQSTOP
        \stopformula

        Additionally: use ln(qtot) in state vector to enforce positivity and
        better error characteristics.

        \cite[Hewison2006]

    \stopsubsection

    \startsubsection[title=Vertical Discretization and Interpolation]

        Smoothing. Compromise between resolution necessary for numerical model
        and actual information content. Inclusion of upper atmosphere or not.

    \stopsubsection

\stopsection


\startsection[title=Atmospheric Extinction]

    Neglecting scattering. Existing absorption models.

    \startsubsection[title=Gaseous Absorption]

        Molecule transitions. Oxygen, Water vapor. What about nitrogen or CO2?
        Included in Liebe Model or negligible? I think Hewison had nitrogen as
        well.

    \stopsubsection

    \startsubsection[title=Liquid Water Extinction]

        Rayleigh-approximation. Problems during precipitation due to
        scattering.

    \stopsubsection

    \startsubsection[title=Fast Absorption Prediction]

        Fitting a polynomial to the absorption at a specific line to speed up
        calculations and simplify derivatives. Evaluate FAP performance for
        each line of the HATPRO.

    \stopsubsection

\stopsection


\startsection[title=Calculating the Jacobian]

    Weighting functions/averaging kernels. Truncation of the jacobian at upper
    levels.

    \startsubsection[title=Finite Differences]

        Can be used with every model. Problems with selection of perturbations,
        non-linearity.
        
    \stopsubsection

    \startsubsection[title=Symbolic Differentiation]

        Non-differentiable parameterizations (LWC).

    \stopsubsection

    \startsubsection[title=Automatic Differentiation]

        Computation graphs. Backwards mode automatic differentiation.
        
    \stopsubsection

\stopsection


\startsection[title=A Numerical Model]

    Custom implementation of a line-by-line model with a mix of automatic and
    symbolic differentiation to calculate accurate and fast jacobians.
    Specializing in one model per frequency allows additional optimizations
    and optimizied FAPs.

    \startsubsection[title=Implementation]

        Custom implementation of Liebe gas absorption, Turner et al. cloud
        absorption.

    \stopsubsection

    \startsubsection[title=Comparison with MonoRTM]

        Introduce MonoRTM wrapper. Statistical comparison of brightness
        temperatures (variablility and bias) as well as jacobians for some
        radiosonde subset.

    \stopsubsection

    \startsubsection[title={Characterizing Errors},reference={ch:rtm_errors}]

        Building part of the forward model covariance matrix.

    \stopsubsection

\stopsection

