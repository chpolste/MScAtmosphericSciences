Information about the vertical state of the atmosphere is of high value
in many areas of meteorology. Vertical profiles of temperature and humidity are
used to determine atmospheric stability and the state of the boundary layer,
have a history of use in day-to-day weather forecasting and are routinely
assimilated into numerical weather prediction models. Currently, in-situ
balloon soundings (radiosoundings), aircraft measurements and remote sensing
data from radar or space-bourne instruments are main sources of information
about the vertical state of the atmosphere
\cite[alternative=authoryears,left={(e.g. }][Brousseau2014]. In recent years,
ground-based radiometers have been established as an additional source of
vertical information about the atmosphere. These passive remote sensing
instruments have been used for decades in research and are increasingly tested
for operational use. Additionally projects and user networks specialized on
microwave radiometers like the Atmospheric Radiation Measurement Program
\cite[authoryears][Cadeddu2013] and MWRnet
(\hyphenatedurl{http://cetemps.aquila.infn.it/mwrnet/}\footnote{last accessed
2016-08-04}) have emerged.


\startsection[title={Properties Ground-based Microwave Radiometers}]

    Measurements from a ground-based microwave radiometers (MWR) can be used to
    determine vertical profiles of temperature, humidity and cloud liquid water
    content \cite[authoryears][Westwater2004] depending on the choice of
    channels which are observed by the instrument. Radiative transfer in the
    microwave region is dominated by absorption and emission of oxygen, water
    vapor and liquid water droplets. Cloud absorption weak enough to allow
    penetration of microwave radiation therefore it is possible to perform
    radiometer measurements even during clouds, giving MWRs an advantage over
    infrared radiometers which are used for similar purposes
    \cite[authoryears][Westwater2004,Kadygrov2015]. The effect of ice particles
    on microwave radiation is negligible, but scattering caused by
    precipitation is too strong to be neglected and retrievals lose accuracy
    (REF).
    
    A radiometer is a passive remote sensing instrument. It can therefore
    be used in settlements \cite[authoryears][Kadygrov2015] and generally
    requires less maintenence and upkeep cost than an active remote sensor
    \cite[authoryears][Guldner2001]. In the long term it is also much
    cheaper than periodically launching balloon-bourne in-situ instruments.
    A disadvantage of being a passive instrument is the need for regular
    calibration \cite[authoryears][Westwater2004,Kadygrov2015]. Common
    calibration methods for microwave radiometers are tipping curve calibration
    and the use of an internal reference target. Both methods only require a
    rotatable antenna, can be automatically carried out by the instrument and
    are usually complemented by less frequent manual calibrations with a
    liquid nitrogen target. The calibration procedure of MWRs overall is well
    understood \cite[authoryears][Cimini2004] and does not affect suitabiliy
    for deployment in a long-term unattended mode
    \cite[authoryears][Westwater2004,Cimini2006].

    It is instructive to compare the capabilities of a ground-based MWR to
    those of other instruments. Radiosondes are usually treated as a reference
    measurement of the thermodynamic atmospheric state due to their relatively
    high sensor accuracy and fine vertical resolution.

    Less representativity error than radiosonde due to spatial and temporal
    averaging \cite[Westwater2004] but lower resolution mainly at higher
    altitudes \cite[Cadeddu2002,Lohnert2004], and problems with inhomogeneity
    of atmosphere especially when elevation scanning is used
    \cite[Cimini2006,Crewell2007,Guldner2013].

    All three sources of measurement suffer from resolution issues: Aircraft
    soundings are available with relatively high frequency but only where
    airports exist.
    Radiosondes are also quite sparse spatially, but are also
    launched only one to four times a day due to high costs of equipment. They
    have the smallest error of all mentioned instruments but suffer from drift
    and locality of measurement.
    Remote sensing instruments of satellites provide area averaged profiles
    with generally limited vertical resolution and have largest errors near the
    surface which is a problem for boundary layer research. While the temporal
    resolution of space-bourne instruments is good, vertical profiles usually
    stem from satellites in polar orbits resulting in large area coverage but
    long times until a measurement is made at the same location again.
    Polar orbiting satellites have global coverage but bad temporal resolution
    at a fixed location and coarse vertical resolution in BL
    \cite[Peckham2000,Sanchez2013].  Ground-based: best resolution in boundary
    layer whereas satellites are better at height (opacity of atmosphere).

\stopsection


\startsection[title={A Short Review of Research and Applications}]

    One of the earliest authors that used ground-based microwave radiometers to
    determine tropospheric temperature profiles is \cite[Westwater1972]. Among
    other publications, the same author has published a review paper focusing
    on remote sensing by ground-based sensors in the microwave and
    millimeter-wave region of the spectrum together with colleages that
    appeared in two different forms \cite[authoryears][Westwater2004,Westwater2005].
    Aside from the physical basics and aspects of instrument construction, the
    authors also discuss derivable thermodynamic quantities and give an
    overview of retrieval techniques. A extensive and general review of remote
    sensing of the lower troposphere including a section about radiometers has
    recently been given by \cite[Wulfmeyer2015]. Retrieval techniques are the
    focus of a paper by \cite[Turner2013]\footnote{This paper has yet to be
    published but can be obtained from
    \hyphenatedurl{http://www.nssl.noaa.gov/users/dturner/public_html/metr5970/retrieval_uncertainty_paper.v4.pdf}}
    who include example applications and recent developments.

    Mobile and relatively compact instrument allowing wide range of
    applications \cite[Kadygrov2015].

    Although not all papers deal with temperature/humidity and some use
    infrared radiation instead of microwaves, the methods of retrieval are the
    same and all papers are considered here when discussing methodology.
    Since the principles and techniques for all are the same, references are
    mixed and include liquid retrievals as well.

    Understanding of BL processes essential for convective model development
    \cite[Martinet2015]. But BL is undersampled part of atmosphere despite
    being important \cite[Cimini2015].  Temporal resolution of seconds, well
    suited for boundary layer research \cite[Cimini2006].

    Numerical models alone unable to fill data gap at mesoscale
    \cite[Sanchez2013].

    Assimilation into forecast models promising \cite[Westwater2004],
    experiments with direct assimilation or radiometer measurements
    \cite[Cimini2006] (check this ref and add RTTOV-GB paper?) and indirect
    assimilation of retrieved profiles \cite[Martinet2015] have been done with
    positive results \cite[Cimini2014]. Currently MWR are not included into
    operational assimilation schemes yet \cite[Lohnert2012].

    Severe weather forecasting is highly dependent on accurate vertical
    profiles \cite[Lohnert2012] and \cite[Cimini2015] (and chinese?) have
    looked at using radiometers to derive severe weather forecasting indices.
    Additonally use at airports to asses fog and stratus development
    \cite[Hewison2004] (chinese also use radiometer in hong kong).

    Temperature retrievals generally work better than humidity reasons
    partially due to microwave absorption properties of atmosphere
    \cite[Lohnert2004,Lohnert2012,Xu2014].

    Reliable first guess in combinead approaches \cite[Cimini2006], retrieval
    methodology extensible in order to incorporate additional measurements.
    Example: integrated profiling technique by \cite[Lohnert2004]. MWR together
    with cloud radar, radiosondes, surface obs and cloud model.

\stopsection


\startsection[title={Prior Work in Innsbruck}]

    The Institute of Atmospheric and Cryospheric Sciences at the University of
    Innsbruck operates a passive microwave radiometer as part of the Innsbruck
    Box project, whose objective it is to study boundary layer dynamics in
    complex terrain. Consequently there have been efforts to retrieve
    temperature and humidity profiles from brightness temperature measurements
    of the radiometer. Research in Innsbruck has been carried out by
    \cite[Massaro2013] in his master's thesis and a subseqent publication
    \cite[authoryears][Massaro2015], followed by a contribution earlier this
    year by \cite[Meyer2016].
    
    Retrievals for the Innsbruck site have been done so far by linear
    regression techniques with a focus on retrieval performance of profiles
    with a surface-based or elevated temperature inversion and the inclusion of
    regressors other than brightness temperatures in the retrieval process. The
    desire to include additional information stems from Innsbruck's location in
    an Alpine valley at an altitude of approximately 600 m, surrounded by
    mountains of over 2000 m height. This complex terrain provides
    a challenging environment for radiometer retrievals as the boundary layer
    evolution is strongly affected by the topography. But the location in the
    mountains also provides data not available in flat regions: weather
    stations exist at different heights in the mountains, providing in-situ
    profiles of the inner-Alpine atmosphere. Even though these measurements
    are affected by surface layer effects they contain valuable information
    about the state of the lower troposphere. Regression techniques allow easy
    integration of such information into a retrieval. The relationship between
    the valley atmosphere measured by the radiometer and surface layer
    measurements obtained from weather stations does not have to be explicitly
    known but is automatically inferred by the regression.

    \cite[Massaro2015] found that the linear regression retrieval performance
    in Innsbruck is comparable performance to flat regions. They found
    significant improvement in the accuracy of the temperature profile at
    heights where additional information are included and for specialized
    retrievals, which categorize the atmospheric state e.g. by stability. Their
    biggest problem remained the accurate retrieval of elevated temperature
    inversions. \cite[Meyer2016] expanded on these results, taking into account
    a greater number of surface observations from various heights and distances
    from the radiometer. The inclusion of data from these stations combined
    with specialized retrievals allowed even the retrieval of elevated
    temperature inversions with satisfying accuracy.

\stopsection


\startsection[title={Objectives}]

    In this thesis the retrieval problem is approached from a Bayesian
    perspective. One of the major benefits of the Bayesian approach is the
    assessment of uncertainty of a retrieved profile. This applies to
    retrievals using linear regression but more importantly allows the
    integration of a radiative transfer model into the retrieval yielding
    the well established optimal estimation technique aka 1D-VAR.

    Considering the good results of the prior work in Innsbruck using linear
    regression, there seems to be little incentive to fundamentally change the
    retrieval method. But there are aspects of regression techniques that are
    disadvantageous. Since regression is a purely statistical method any
    knowledge of the underlying physics is - if at all - only implicitly
    included. Introducing additional regressors such as surface temperature
    measurements provide no hard constraints for the retrieval. Similarly
    there is no explicit connection between adjacent atmospheric layers because
    in training the problem decouples into individual regressions for each
    layer. Physically reasonable behavior of retrieved profiles (as usually
    observed in practice) is a consequence of a representative climatology.
    But there is always the possibility of a situation appearing that is far
    away from the climatological range leading to poor regression performance.
    Innsbruck for example has the problem that radiosondes are only launched at
    night, which likely leads to worse retrieval performance during the day
    (without a reasonably sized test dataset of daytime radiosondes this is not
    easy to verify).  Other locations might not even have a training set of
    radiosonde profiles available for training. Model forecasts can be used for
    such locations (CITE THIS) but these are affected by problems associated
    with NWP.

    Optimal estimation techniques following a Bayesian approach improve on
    these shortcomings of regression. Uncertainty assessment is natural to all
    Bayesian methods since entire distributions are propagated through the
    retrieval instead of just values. Physical knowledge, e.g. from a radiative
    transfer model, is explicitly incorporated during retrieval making it
    possible to accurately obtain profiles in situations not covered by
    a climatology. Adding additional instruments to the retrieval scheme seems
    mathematically more elegant (EHHH) than in regression methods and is
    generally more flexible (CITE IPT PAPER?). These benefits come at a cost.
    The radiative transfer model only works with profiles of temperature,
    humidity and cloud liquid water making retrievals of individual variables
    impossible. Computational costs are higher, although this concern is
    becoming increasingly irrelevant with faster models and computers. Even
    though optimal estimation methods do not require a climatology, the data
    that is used must have well characterized errors and known systematic
    biases. This makes the setup much more complicated than that of
    a regression method.

    The goal of this thesis is to provide such a setup for profile retrievals
    in Innsbruck paving the way to operational retrievals useful for boundary
    layer research and weather forecasting. It is an exploratory study as
    variational techniques have never been used for retrievals in Innsbruck.
    Most of the theoretical background of the method and radiative transfer
    is shown here in order to build a deep understanding of the method and all
    its assumptions and implications. Results of a first retrieval
    implementation are shown and compared to regression retrievals. Finally,
    shortcomings of the presented work and ideas for future research are
    given.

\stopsection

