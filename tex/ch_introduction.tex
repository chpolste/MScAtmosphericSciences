Information about the vertical state of the atmosphere is of high value
in many areas of meteorology. Vertical profiles of temperature and humidity are
used to determine atmospheric stability and the state of the boundary layer,
have a history of use in day-to-day weather forecasting and are routinely
assimilated into numerical weather prediction models. In-situ balloon soundings
(radiosoundings), aircraft measurements and remote sensing data from radar or
space-borne instruments are the main sources of information about the vertical
state of the atmosphere today \cite[alternative=authoryears,left={(e.g.
}][Brousseau2014].

The microwave radiometer, a passive remote sensing instrument, has been used
for decades in research to derive vertical thermodynamic information of the
troposphere and is increasingly tested in operational scenarios where such
information is useful. Ground-based microwave radiometers are available
commercially and projects specialized on these instruments like the Atmospheric
Radiation Measurement Program \cite[authoryears][Cadeddu2013] and MWRnet
(\hyphenatedurl{http://cetemps.aquila.infn.it/mwrnet/} have emerged. The
improvement of methods for the retrieval of vertical atmospheric profiles is
a topic of current research. This thesis contributes to such efforts at the
University of Innsbruck.

\startsection[title={Properties and Applications of Microwave Radiometers}]

    Measurements from a ground-based microwave radiometer (MWR) can be used to
    determine vertical profiles of temperature, humidity and cloud liquid water
    content \cite[authoryears][Westwater2004].  In practice it has been
    found that temperature retrievals generally achieve higher accuracy than
    humidity retrievals, partially due to the properties of microwave radiative
    transfer in the atmosphere which is governed by absorption and emission of
    oxygen, water vapor and liquid water droplets
    \cite[authoryears][Lohnert2004,Xu2014].
    The absorption of microwave radiation by cloud droplets is sufficiently
    weak to allow penetration of the cloud and it is therefore possible to
    perform radiometer measurements even during cloudy skies, giving MWRs an
    advantage over infrared radiometers which are used for similar purposes but
    cannot see through clouds \cite[authoryears][Westwater2004,Kadygrov2015].
    The effect of ice particles on microwave radiation is negligible, but
    scattering caused by precipitation is strong and causes retrievals to lose
    accuracy.
    
    A radiometer is a passive remote sensing instrument. It can therefore
    be used in settlements \cite[authoryears][Kadygrov2015] and generally
    requires less maintenance and than active remote sensors
    \cite[authoryears][Guldner2001]. It is much cheaper to operate than regular
    launches of balloon-borne in-situ instruments and provides a much higher
    temporal resolution. A disadvantage of being a passive instrument is the
    need for regular calibration \cite[authoryears][Westwater2004,Kadygrov2015].
    Common calibration methods for microwave radiometers are tipping curve
    method and calibration with an internal black-body reference target. Both
    methods only require a rotatable antenna, can be automatically carried out
    by the instrument and are typically complemented by less frequent manual
    calibrations with a liquid nitrogen target. The calibration procedure of
    MWRs overall is well understood \cite[authoryears][Cimini2004] and does not
    affect suitability for deployment in a long-term unattended mode
    \cite[authoryears][Westwater2004,Cimini2006].

    It is instructive to compare the capabilities of various instruments
    commonly used to obtain the same atmospheric variables retrievable from
    a ground-based MWR. Radiosondes are usually treated as a reference
    measurement of the thermodynamic atmospheric state due to their high sensor
    accuracy and fine vertical resolution. However, balloons experience
    horizontal drift during their ascent and need a substantial amount of time
    to reach the tropopause which negatively affects their representativity
    for a given location. The relatively high costs of sensors and balloons
    causes radiosonde launch sites to be distributed only sparsely and launches
    typically occur only one to four times a day. This temporal resolution is
    insufficient to appropriately capture the daily boundary layer evolution.
    Aircraft soundings are available with much better temporal resolution but
    only where airports exist and they have similar representativity issues as
    radiosondes due to the horizontal displacement of the aircraft path during
    ascent and descent. Remote sensing instruments provide volume averaged profiles
    which have a better representativity of the atmospheric state at the cost
    of vertical resolution \cite[authoryears][Westwater2004].  Remote sensors
    can achieve high temporal resolution depending on the need for averaging to
    reduce instrument noise. Instruments on a satellite in a polar orbit have
    global coverage but pass a fixed location only after long intervals twice
    and temperature and humidity sensors on satellites provide only coarse
    vertical resolution in the boundary layer
    \cite[authoryears][Peckham2000,Sanchez2013]. The resolution of
    a ground-based microwave radiometer is highest in the boundary layer and
    decreases with altitude \cite[authoryears][Cadeddu2002,Lohnert2004].
    Low-level resolution can additionally be increased by scanning in
    off-zenith directions if the atmosphere is weakly inhomogeneous in the
    horizontal \cite[authoryears][Cimini2006,Crewell2007,Guldner2013].
    Together with their high temporal resolution this makes ground-based MWRs
    well suited for boundary layer research \cite[authoryears][Cimini2006]. The
    spatial coverage of a network of ground-based microwave radiometers
    obviously cannot compete with the coverage of a satellite in a polar orbit
    but the instruments are mobile, inexpensive and relatively compact,
    allowing a wide range of applications \cite[authoryears][Kadygrov2015].

    There are attempts to counter the lack of vertical information in the
    boundary layer with numerical weather prediction (NWP) models but so far
    mesoscale models alone have not been able to fill this data gap
    \cite[authoryears][Sanchez2013]. In the contrary, the understanding of
    boundary layer processes is essential for the development of
    convective-scale models \cite[authoryears][Martinet2015]. Current research
    is concerned with the assimilation of observations from ground-based
    microwave radiometers into NWP models which is not being done operationally
    yet \cite[authoryears][Lohnert2012]. An experiment with assimilation of
    temperature and humidity profiles retrieved from 13 MWRs by
    \cite[Cimini2014] resulted in a neutral-to-positive impact on forecast
    skills. Recently a numerical radiative transfer model has been developed by
    \cite[DeAngelis2016] with the goal of direct assimilation of radiometer
    observations.

    Not only numerical weather prediction benefits from additional
    information about the vertical state of the atmosphere, but also other
    forms of forecasts. Most prominent are short-term severe weather forecasts
    which strongly depend on accurate vertical profiles of temperature, humidity
    and wind \cite[authoryears][Lohnert2012]. \cite[Madhulatha2013,Cimini2015]
    used radiometer observations to derive indices of severe convective
    activity and fog. \cite[Chan2013] analyzed the fluctuations of brightness
    temperatures measured by a MWR to assess low-level wind shear, crucial for
    the safe operation of an airport. Such information is also valuable for the
    safety of nuclear power plants and the assessment of local pollution
    propagation \cite[authoryears][Westwater1972,Kadygrov2015].
    
    A few notable publications shall be mentioned to conclude this section.
    A review of remote sensing by ground-based sensors in the microwave and
    millimeter-wave region was given by \cite[Westwater2004]. Aside from the
    physical basics and aspects of instrument construction, the authors also
    discuss derivable thermodynamic variables and give an overview of
    retrieval techniques. An extensive and general review of remote sensing of
    the lower troposphere including a section about radiometers has recently
    been given by \cite[Wulfmeyer2015]. Retrieval techniques are the focus of
    a yet unpublished article by \cite[Turner2013]\footnote{ which can be obtained from
    \hyphenatedurl{http://www.nssl.noaa.gov/users/dturner/public_html/metr5970/retrieval_uncertainty_paper.v4.pdf}}.
    The authors show example applications and discuss recent developments.

\stopsection


\startsection[title={Prior Work in Innsbruck}]

    The Institute of Atmospheric and Cryospheric Sciences at the University of
    Innsbruck operates a passive microwave radiometer as part of the Innsbruck
    Box project, whose objective is to study boundary layer dynamics in complex
    terrain. Consequently, there have been efforts to retrieve temperature and
    humidity profiles from brightness temperature measurements of the
    radiometer. Research in Innsbruck has been carried out by
    \cite[Massaro2013] in his master's thesis and a subsequent publication
    \cite[authoryears][Massaro2015], followed by a contribution earlier this
    year by \cite[Meyer2016].
    
    Retrievals for the Innsbruck site have been done so far with linear
    regression techniques and a focus on retrieval performance during
    surface-based or elevated temperature inversions. The inclusion of
    regressors other than brightness temperatures in the retrieval process has
    been extensively studied. The desire to include additional information
    stems from Innsbruck's location in an Alpine valley, surrounded by
    mountains of over 2000 m height. This complex terrain provides
    a challenging environment for radiometer retrievals as the boundary layer
    evolution is strongly affected by the topography. But the location in the
    mountains also provides data not available in flat regions: weather
    stations exist at different heights in the mountains, providing in-situ
    profiles of the inner-Alpine atmosphere. Even though these measurements
    are affected by surface layer effects they contain valuable information
    about the state of the lower troposphere. Regression techniques allow easy
    integration of such information into a retrieval.

    \cite[Massaro2015] found that the linear regression retrieval performance
    in Innsbruck is comparable performance to that of radiometers in flat
    regions. They found significant improvement in the accuracy of the
    temperature profile at heights where in-situ information is included and
    for specialized retrievals, categorizing retrievals for example by
    atmospheric stability and applying specifically trained regression models.
    The accurate retrieval of elevated temperature inversions remained
    a challenge. \cite[Meyer2016] expanded on these results, taking into
    account a greater number of surface observations from various heights and
    distances from the radiometer. The inclusion of data from these stations
    combined with specialized retrievals allowed even the retrieval of elevated
    temperature inversions with satisfying accuracy.

\stopsection


\startsection[title={Objectives and Outline}]

    In this thesis the retrieval problem is approached from a Bayesian
    perspective. One of the major benefits of the Bayesian approach is the
    assessment of uncertainty of a retrieved profile and the possibility to
    explicitly integrate a radiative transfer model into the retrieval.

    Considering the good results of prior work in Innsbruck with linear
    regression models, there seems to be little incentive to fundamentally change the
    retrieval method. But there are aspects of regression techniques that are
    disadvantageous. They are a purely statistical method and any
    knowledge of the underlying physics of the retrieval problem is only
    considered implicitly. Deriving error estimates for retrieved profiles
    from a regression model is hard.  The physically reasonable behavior of
    retrieved profiles is a consequence of a representative climatology
    but there is always the possibility of a situation appearing far away from
    the climatological range, leading to poor regression performance.
    Innsbruck for example has the problem that radiosondes are only launched at
    night, which likely leads to worse retrieval performance during the day
    (without a reasonably sized test dataset of daytime radiosondes this is not
    easy to verify).  Other locations might not even have a training set of
    radiosonde profiles available for training.

    The optimal estimation technique improves on such shortcomings of
    regression. Uncertainty assessment is natural to all Bayesian methods since
    distributions are propagated through the retrieval instead of just values.
    Physical knowledge from a radiative transfer model is explicitly
    incorporated, allowing accurate retrievals in situations not covered by
    a climatology. Observations from other sensors can be integrated into the
    method \cite[alternative=authoryears,left={(see e.g. },right={ who combined
    information from 5 sources with an optimal estimation method)}][Lohnert2004].
    However computational of retrievals are higher and any data used in the
    retrieval must have well characterized errors and known systematic biases.
    This makes the setup more complicated than that of a regression method.

    The goal of this thesis is to provide an implementation of optimal
    estimation for retrievals of vertical temperature and humidity profiles,
    paving the way to operational retrievals useful for boundary
    layer research and weather forecasting in Innsbruck. The theoretical
    background of the method and radiative transfer is shown here in order to
    build a deep understanding of the method and its assumptions. Results of
    the implementation are shown and compared to retrievals by a regression
    model.

\stopsection

