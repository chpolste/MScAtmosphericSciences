Why radiometer measurements → boundary layer, high time resolution.

Applications: Fundamental ABL-research, nowcasting, NWP assimilation, ....

History of application in Innsbruck: Linear regression techniques, focus on
inclusion of additional information from different heights. Advantage of
location in mountains: vertical profiles of the lower atmosphere can be made
directly with stations along a mountain slope, if surface layer effects are
taken into account. Linear regression allows very easy integration of such
measurements, no explicit consideration of surface layer processes necessary.
But: additional regressors do not constrain the solution, regression training
might find good correlation and put much weight into such an additional
regressor, but the retrieved profile is not explicity bound to the measured
value. Additionally problem of uncertainty assessment, which is addressed here.
Bayesian technique used here improves on these points: uncertainty assessment
of retrieved profile and explicit consideration of measurement errors. Also:
use correlations between adjacent atmospheric layers, while regression
decouples into a separate problem for each layer.

Goals of this thesis: quick literature review, building foundation for future
work in Innsbruck, towards operational retrievals for ABL-research (ibox) and
ertel2 visulalization (mention recently added lidar). Bayesian techniques
require much more effort up front than regression techniques and seem much more
subjective in the choice of error margins etc. but actually all these
assumptions are in a way also included in a regression model, just implicit.
Making the user aware of these assumptions is important for understanding of
the results and might be seen as an advantage even though it means more work.
Bayesian approach is much more powerful than regression, can adapt to
situations not included in the climatology, can include information from other
instruments and NWP, which is especially helpful for the retrieval of
upper-troposphere data.

