Why radiometer measurements → boundary layer, high time resolution.

Applications: Fundamental ABL-research, nowcasting, NWP assimilation, ....


\startsection[title={State of Research}]

    Literature Review, from beginnings of regression to rise of variational
    methods.

\stopsection


\startsection[title={Prior Work and Objectives}]

    The Institute of Atmospheric and Cryospheric Sciences at the University of
    Innsbruck operates a passive microwave radiometer as part of the Innsbruck
    Box project, whose objective it is to study boundary layer dynamics in
    complex terrain. Consequently there have been efforts to retrieve
    temperature and humidity profiles from brightness temperature measurements
    of the radiometer. Research in Innsbruck has been carried out by
    \cite[Massaro2013] in his master's thesis and a subseqent publication
    \cite[authoryears][Massaro2015], followed by a contribution earlier this
    year by \cite[Meyer2016].
    
    Retrievals for the Innsbruck site have been done so far by linear
    regression techniques with a focus on retrieval performance of profiles
    with a surface-based or elevated temperature inversion and the inclusion of
    regressors other than brightness temperatures in the retrieval process. The
    desire to include additional information stems from Innsbruck's location in
    an Alpine valley at an altitude of approximately 600 m, surrounded by
    mountains of over 2000 m height. This complex terrain provides
    a challenging environment for radiometer retrievals as the boundary layer
    evolution is strongly affected by the topography. But the location in the
    mountains also provides data not available in flat regions: weather
    stations exist at different heights in the mountains, providing in-situ
    profiles of the inner-Alpine atmosphere. Even though these measurements
    are affected by surface layer effects they contain valuable information
    about the state of the lower troposphere. Regression techniques allow easy
    integration of such information into a retrieval. The relationship between
    the valley atmosphere measured by the radiometer and surface layer
    measurements obtained from weather stations does not have to be explicitly
    known but is automatically inferred by the regression.

    \cite[Massaro2015] found that the linear regression retrieval performance
    in Innsbruck is comparable performance to flat regions. They found
    significant improvement in the accuracy of the temperature profile at
    heights where additional information are included and for specialized
    retrievals, which categorize the atmospheric state e.g. by stability. Their
    biggest problem remained the accurate retrieval of elevated temperature
    inversions. \cite[Meyer2016] expanded on these results, taking into account
    a greater number of surface observations from various heights and distances
    from the radiometer. The inclusion of data from these stations combined
    with specialized retrievals allowed even the retrieval of elevated
    temperature inversions with satisfying accuracy.

    Considering these results, there seem to be little incentive to
    fundamentally change the retrieval method. But there are aspects of
    regression techniques that are disadvantageous.

    No quantified uncertainty. Additional regressors do not constrain the
    solution, regression training might find good correlation and put much
    weight into such an additional regressor, but the retrieved profile is not
    explicity bound to the measured value. Additionally problem of uncertainty
    assessment, which is addressed here.  Bayesian technique used here improves
    on these points: uncertainty assessment of retrieved profile and explicit
    consideration of measurement errors. Also: use correlations between
    adjacent atmospheric layers, while regression decouples into a separate
    problem for each layer. Non-obvious integration of information such as
    cloud base height. No guaranteed consistency between temperature, humidity
    and brightness temperatures. Only what's contained in the climatology can
    be reproduced, this is especially problematic in Innsbruck where
    radiosondes are only lauched once a day at night. Known bad performance in
    upper atmosphere. Many implicit assumptions in regression model. Complete
    disregard of physical knowledge, only statisitical relationship.

    TODO: Height attribution of clouds in the valley is possible without
    ceilometers or cloud radar due to terrain markers, most NWP models resolve
    the terrain insufficiently. Humidity too localized for integration of
    surface measurements?

    Goals of this thesis: building foundation for future work in Innsbruck,
    towards operational retrievals for ABL-research (ibox) and ertel2
    visualization (mention recently added lidar). Bayesian techniques require
    much more effort up front than regression techniques and seem much more
    subjective in the choice of error margins etc. but actually all these
    assumptions are in a way also included in a regression model, just
    implicitly. Making the user aware of these assumptions is important for
    understanding of the results and might be seen as an advantage even though
    it means more work. Bayesian approach is much more powerful than
    regression, can adapt to situations not included in the climatology, can
    include information from other instruments and NWP, which is especially
    helpful for the retrieval of upper-troposphere data. Generally the Bayesian
    approach is mathematically cleaner.

\stopsection

