Radiosondes, aircraft measurements and satellites are main sources of vertical
information about atmospheric state. Such information is of high values
for many areas of meteorology, from boundary layer research to numerical
weather prediction. Vertical profiles of temperature and humidity are a main
indicator for the state of the atmospheric boundary layer, they are used in
day-to-day weather forecasting and are routinely assimilated into numerical
weather prediction models. All three sources of measurement suffer from
resolution issues: Aircraft soundings are available with relatively high
frequency but only where airports exist. Radiosondes are also quite sparse
spatially, but are also launched only one to four times a day due to high
costs of equipment. They have the smallest error of all mentioned instruments
but suffer from drift and locality of measurement. Remote sensing instruments
of satellites provide area averaged profiles with generally limited vertical
resolution and have largest errors near the surface which is a problem for
boundary layer research. While the temporal resolution of space-bourne
instruments is good, vertical profiles usually stem from satellites in
polar orbits resulting in large area coverage but long times until
a measurement is made at the same location again.

From the desire for a ground-based instrument with lower operating costs than
balloon soundings and high temporal resolution came microwave radiometers...

Passive instrument, very high time resolution, little maintenance necessary,
fairly cheap components, portable. Ground based means best vertical resolution
in boundary layer. Can see through clouds in contrast to IR instruments but
problems in rain due to scattering. Aside from scattering issues in rain only
few atmospheric constituents to consider in radiative transfer.

TODO: expand a little, put 1 or 2 citations in here.


\startsection[title={State of Research}]

    Literature Review, from beginnings of regression to rise of variational
    methods.

\stopsection


\startsection[title={Prior Work in Innsbruck}]

    The Institute of Atmospheric and Cryospheric Sciences at the University of
    Innsbruck operates a passive microwave radiometer as part of the Innsbruck
    Box project, whose objective it is to study boundary layer dynamics in
    complex terrain. Consequently there have been efforts to retrieve
    temperature and humidity profiles from brightness temperature measurements
    of the radiometer. Research in Innsbruck has been carried out by
    \cite[Massaro2013] in his master's thesis and a subseqent publication
    \cite[authoryears][Massaro2015], followed by a contribution earlier this
    year by \cite[Meyer2016].
    
    Retrievals for the Innsbruck site have been done so far by linear
    regression techniques with a focus on retrieval performance of profiles
    with a surface-based or elevated temperature inversion and the inclusion of
    regressors other than brightness temperatures in the retrieval process. The
    desire to include additional information stems from Innsbruck's location in
    an Alpine valley at an altitude of approximately 600 m, surrounded by
    mountains of over 2000 m height. This complex terrain provides
    a challenging environment for radiometer retrievals as the boundary layer
    evolution is strongly affected by the topography. But the location in the
    mountains also provides data not available in flat regions: weather
    stations exist at different heights in the mountains, providing in-situ
    profiles of the inner-Alpine atmosphere. Even though these measurements
    are affected by surface layer effects they contain valuable information
    about the state of the lower troposphere. Regression techniques allow easy
    integration of such information into a retrieval. The relationship between
    the valley atmosphere measured by the radiometer and surface layer
    measurements obtained from weather stations does not have to be explicitly
    known but is automatically inferred by the regression.

    \cite[Massaro2015] found that the linear regression retrieval performance
    in Innsbruck is comparable performance to flat regions. They found
    significant improvement in the accuracy of the temperature profile at
    heights where additional information are included and for specialized
    retrievals, which categorize the atmospheric state e.g. by stability. Their
    biggest problem remained the accurate retrieval of elevated temperature
    inversions. \cite[Meyer2016] expanded on these results, taking into account
    a greater number of surface observations from various heights and distances
    from the radiometer. The inclusion of data from these stations combined
    with specialized retrievals allowed even the retrieval of elevated
    temperature inversions with satisfying accuracy.

\stopsection


\startsection[title={Objectives}]

    In this thesis the retrieval problem is approached from a Bayesian
    perspective. One of the major benefits of the Bayesian approach is the
    assessment of uncertainty of a retrieved profile. This applies to
    retrievals using linear regression but more importantly allows the
    integration of a radiative transfer model into the retrieval yielding
    the well established optimal estimation technique aka 1D-VAR.

    Considering the good results of the prior work in Innsbruck using linear
    regression, there seems to be little incentive to fundamentally change the
    retrieval method. But there are aspects of regression techniques that are
    disadvantageous. Since regression is a purely statistical method any
    knowledge of the underlying physics is - if at all - only implicitly
    included. Introducing additional regressors such as surface temperature
    measurements provide no hard constraints for the retrieval. Similarly
    there is no explicit connection between adjacent atmospheric layers because
    in training the problem decouples into individual regressions for each
    layer. Physically reasonable behavior of retrieved profiles (as usually
    observed in practice) is a consequence of a representative climatology.
    But there is always the possibility of a situation appearing that is far
    away from the climatological range leading to poor regression performance.
    Innsbruck for example has the problem that radiosondes are only launched at
    night, which likely leads to worse retrieval performance during the day
    (without a reasonably sized test dataset of daytime radiosondes this is not
    easy to verify).  Other locations might not even have a training set of
    radiosonde profiles available for training. Model forecasts can be used for
    such locations (CITE THIS) but these are affected by problems associated
    with NWP.

    Optimal estimation techniques following a Bayesian approach improve on
    these shortcomings of regression. Uncertainty assessment is natural to all
    Bayesian methods since entire distributions are propagated through the
    retrieval instead of just values. Physical knowledge, e.g. from a radiative
    transfer model, is explicitly incorporated during retrieval making it
    possible to accurately obtain profiles in situations not covered by
    a climatology. Adding additional instruments to the retrieval scheme seems
    mathematically more elegant (EHHH) than in regression methods and is
    generally more flexible (CITE IPT PAPER?). These benefits come at a cost.
    The radiative transfer model only works with profiles of temperature,
    humidity and cloud liquid water making retrievals of individual variables
    impossible. Computational costs are higher, although this concern is
    becoming increasingly irrelevant with faster models and computers. Even
    though optimal estimation methods do not require a climatology, the data
    that is used must have well characterized errors and known systematic
    biases. This makes the setup much more complicated than that of
    a regression method.

    The goal of this thesis is to provide such a setup for profile retrievals
    in Innsbruck paving the way to operational retrievals useful for boundary
    layer research and weather forecasting. It is an exploratory study as
    variational techniques have never been used for retrievals in Innsbruck.
    Most of the theoretical background of the method and radiative transfer
    is shown here in order to build a deep understanding of the method and all
    its assumptions and implications. Results of a first retrieval
    implementation are shown and compared to regression retrievals. Finally,
    shortcomings of the presented work and ideas for future research are
    given.

\stopsection

