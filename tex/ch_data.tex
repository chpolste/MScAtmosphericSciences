The optimal estimation and linear regression methods are used for retrievals of
temperature and humidity profiles based on data collected in Innsbruck.
A radiosonde climatology provides reference profiles for the evaluation of
the performance of the retrieval techniques based on simulated brightness
temperatures and actual radiometer data. Additional prior information is
provided by forecasts and analyses of a regional numerical weather prediction
model. Figure \in[fig:map_ibk] gives an overview of the Innsbruck region and
the locations of these data sources.


\startsection[title={Radiosonde Climatology},reference=ch:rasoclim]

    Radiosondes are launched regularly once per day from Innsbruck airport
    at the location of marker 1 in Figure \in[fig:map_ibk]. Launch time is
    usually between 01:00 and 04:00 UTC, which corresponds to 02:00 to 05:00
    local time in winter and 03:00 to 06:00 local time during daylight saving
    time. Here, the climatological dataset of temperature, humidity and
    pressure spanning the years 1999 to 2005 and 2009 to 2012, previously used
    by \cite[Massaro2013] and \cite[Meyer2016], is extended with profiles
    obtained between 2013 and 2016.  In total, data from 3561 radiosonde are
    available which are separated into a training data set of 3296 profiles and
    a test data set of 265 profiles which contains all data between February
    2015 and January 2016. This partitioning results in the largest overlap of
    test data with other data sources.

    Because liquid water content is not measured by any instrument during
    a balloon sounding, cloud water content is calculated by the empirical
    cloud model of \cite[Karstens1994] based on humidity and temperature
    information. This model is frequently used for microwave radiometer
    retrieval studies, e.g. by \cite[Lohnert2012,Martinet2015]. After
    application of the model, 52 \% of all training profiles and 53 \% of all
    test profiles have at least one layer with cloud water content larger than
    zero and are therefore categorized as cloudy.

    The profiles of pressure, temperature, humidity and liquid water content
    are linearly interpolated to the 50 levels of the retrieval grid. Figure
    \in[fig:raso_prior] shows the climatology's mean profile of temperature and
    total water content together with intervals of ±1 and ±2 standard
    deviation. Temperature variability is almost constant throughout the
    troposphere, while water variability is greater near the surface and
    vanishes at the mean height of the tropopause. The latter is due to the
    small water vapor content in the upper atmosphere. Total water content
    uncertainty intervals are asymmetric because the underlying distribution is
    a Gaussian in logarithmic space (see section \in[ch:statevector]).

    Whenever the accuracy of a retrieval method is evaluated in subsequent
    sections, these radiosonde data are the reference that retrieved profiles
    are compared to. It must be noted that radiosondes provide point
    measurements, are affected by horizontal drift and take a substantial amout
    of time to reach their final height. These properties of balloon-bourne
    sensors raise concerns regarding the representativity of the measured
    vertical profiles for a fixed location. Only 4 \% of all radiosonde
    profiles were obtained during day time. However, because the test data set
    only consists of nighttime soundings this bias of climatology should not be
    noticeable in the subsequent performance assessments of retrieval methods
    based on the training set of radiosonde profiles.


\stopsection

\placefigure[top][fig:map_ibk]
    {Data locations in the Inn-valley around Innsbruck. 1: Innsbruck
    Airport (577 m), radiosonde launch location. 2: HATPRO location with
    approximate elevation scanning direction (612 m). 3: COSMO-7 model grid
    point. Map source: http://www.basemap.at}
    {\externalfigure[map][width=\textwidth]}


\startsection[title=COSMO-7 Simulated Soundings]

    COSMO-7 is a regional weather prediction model for western and central
    Europe operated by MeteoSwiss. Its terrain-following grid has a
    horizontal mesh size of 6.6 km and 60 vertical levels up to 21453
    m altitude. Vertical profiles of temperature, humidity and cloud water
    content for a gridpoint near Innsbruck (marker 3 in Figure
    \in[fig:map_ibk]) have been obtained for the timespan 2015-02-06 to
    2016-01-20 from forecasts initiated at 00 UTC. Liquid water content
    $\QLIQ$ is extracted from the total cloud water content $\QCLOUD$ based on
    the temperature $T$ of the cloud according to the formula

    \startformula
        \QLIQ = \QCLOUD \startcases
            \NC 0 \MC 233.15 \, \KELVIN \le \TEMP \NR
            \NC \frac{\TEMP - 233.15 \, \KELVIN}{40 \, \KELVIN}
                \MC 233.15 \, \KELVIN \lt \TEMP \lt 273.15 \, \KELVIN \EQSTOP \NR
            \NC 1 \MC 273.15 \, \KELVIN \le \TEMP \NR
        \stopcases
    \stopformula

    Forecasts are available in 6 h steps and up to +30 h. Profiles are linearly
    interpolated to match any given time of retrieval between two lead times.

\placefigure[top][fig:raso_prior]
        {The radiosonde climatology of temperature (left) and total water
        content (vapor and liquid water) derived from soundings launched
        at Innsbruck airport. The black lines indicate the climatological
        mean values, grey shadings mark the one (dark) and two (light)
        standard deviation interals obtained from sampling the
        distribution.
        }
        {\externalfigure[raso_prior][width=\textwidth]}

    Because the surface at the gridpoint near Innsbruck is located in the model
    topography more than 400 m too high at an altitude of 1093 m, the COSMO-7
    profiles have to be extrapolated to the true surface height. At first it was
    considered to simply connect the lowest level of the COSMO-7 profile to
    actual surface measurements of temperature and humidity. This however
    disregards information from the boundary layer processes resolved by the
    model which would then be elevated features. Instead, profiles are
    stretched to the ground by replacing the original height grid of COSMO-7
    by one that ends at the same altitude but starts at the true surface.

    The new grid is chosen such that lower levels are streched more than
    upper levels but the overall relative spacing of the model levels is
    approximately preserved. Because this procedure does not take temperature
    gradients into account the resulting profiles likely have a significant
    bias relative to the true atmospheric state. To counter this, the mean
    bias of the model profiles with respect to the radiosondes from Innsbruck
    airport is removed from the dataset. Because there are only 253 times
    where a radiosounding exists parallel to a COSMO-7 forecast this bias is
    determined from the entire data set. An important implication of this
    choice is that COSMO-7 profiles used for the retrieval are not independent
    of the radiosonde reference data set. This inconvenience is accepted in
    favor of a more robust bias determination based on a larger data set.

    In order to use the COSMO-7 forecasts as a prior for optimal estimation
    retrievals a covariance matrix quantifying the model's uncertainty must
    be known. As for the bias, this matrix is calculated from a comparison
    with the reference radiosonde profiles. Figure \in[fig:cosmo7_prior]
    shows the uncertainty range of a model profile attached to a forecast
    interpolated to 2015-09-11 03:48 UTC in the lower half of the troposphere.
    Temperature uncertainty is highest in the lower troposphere, but smaller
    outside the boundary layer. Uncertainty increases above 6 km only in the
    tropopause region. The water content uncertainty is approximately constant
    up to 5 km above which it is decrasing toward zero. The example profile
    indicates that COSMO-7 is able to simulate temperature inversions in the
    lower atmosphere.

    \placefigure[top][fig:cosmo7_prior]
            {A COSMO-7 prior distribution of temperature (left) and total water
            content humidity (right) valid at 2015-02-06 03:02 UTC. The means
            (black lines) are interpolated from lead times +00 and +06 of the
            2015-02-06 00:00 UTC run. The grey shadings show the one (dark) and
            two (light) standard deviation intervals obtained by sampling.
            Shown is the lower half of the troposphere.
            }
            {\externalfigure[cosmo7_prior][width=\textwidth]}

\stopsection


\startsection[title=HATPRO Observations,reference=ch:hatpro]

    The microwave radiometer operated by the The Institute of Atmospheric and
    Cryospheric Sciences at the University of Innsbruck is a Humidity and
    Temperature Profiler (HATPRO), built by the Radiometer Physics GmbH
    \cite[authoryears][Rose2005]. It records brightness temperatures in 14
    channels with a temporal resolution of 1 s. Channels in the K band
    correspond to frequencies of 22.24, 23.04, 23.84, 25.44, 26.24, 27.84 and
    31.40 GHz. Channels in the V band correspond to frequencies of 51.26,
    52.28, 53.86, 54.94, 56.66, 57.30 and 58.00 GHz. These are marked with
    arrows in Figure \in[fig:absorption]. The receiver antenna can be rotated
    to perform elevation scans. Environmental pressure, temperature and
    humidity are additionaly measured by the instrument and it has
    a precipitation sensor which is translated to a rain/no rain flag in the
    output data.

    The radiometer is set up on the roof of a university building at an
    altitude of 612 m above sea level. Figure \in[fig:map_ibk] shows the
    location of the radiometer site in the Inn Valley (marker 2). The distance
    to the radiosonde launch site at Innsbruck airport (marker 1) is
    approximately 2.5 km. The height difference between these sites is 35 m
    with the HATPRO located higher. The elevation scan direction (indicated by
    the triangle) is downvalley, in the opposite direction of the airport.

    Data from the instrument are available in the timespan from August to
    October 2015. In this period, HATPRO was configured to perform boundary
    layer scans every 10 minutes with averaging times of approximately 30 s
    each at zenith angles of 0°, 60°, 70.8°, 75.6°, 78.6°, 81.6°, 83.4°.
    HATPRO observations were matched to 26 radiosonde launches by taking the
    closest boundary layer scan with at most 30 minutes time difference to the
    launch of the radiosonde.

    \placefigure[top][fig:model_bias]
            {Radiative transfer model biases relative to HATPRO observations
            determined from 10 clear sky cases of matching radiometer and
            radiosonde measurements. Compared are simulations of MonoRTM
            \cite[authoryears][Clough2005] and MWRTM (section \in[ch:mwrtm])
            with (MWRTM/FAP) and without a fast absorption predictor.
            }
            {\externalfigure[model_bias][width=\textwidth]}

    Mean and covariance of the error distribution \ineq{optimalest_errpdf} are
    governed by the bias and uncertainty of forward model error and
    instrumental noise. Because no observation-based estimate of the
    instrumental noise was available for the ACINN HATPRO, a radiometric
    accuracy of 0.5 K standard deviation is assumed for all channels. This
    value is likely too high for the instrumental error alone considering that
    \cite[Rose2005] claim at most 0.4 K error for 1 s integration times but
    fluctuations of temperature and humidity in the atmosphere cause
    measurement noise as well. In any case, a conservative estimate of the
    instrumental noise is certainly favorable to an underestimation. The bias
    of radative transfer models with respect to the HATPRO observations is
    determined from clear sky observations matched to radiosonde ascents.
    Unfortunately, only 11 profiles were classified as clear-sky from the
    radiosonde data and webcam pictures of the Inn valley revealed that only 10
    of these profiles were actually cloud free. Bias values calculated from
    this rather poor set of data are shown in Figure \in[fig:model_bias].

    MWRTM has the largest bias values in the K band channels and simulates
    brightness temperatures systematically warmer than measurements. This
    behavior was also found by \cite[Cimini2004] for a model based on the
    \cite[Liebe1993] absorption description. In the V band, MWRTM with and
    without the fast absorption predictor is closer to the HATPRO observations
    than MonoRTM for the transparent channels. Biases of both models are very
    similar and small in the opaque V band channels. The depicted bias values
    are used for the mean of the error distribution \ineq{optimalest_errpdf}
    in optimal estimation retrievals based on HATPRO observations.

\stopsection


\startsection[title={Retrieval Setup}]

    Synthetic brightness temperature data used as input for retrievals
    and the training of regression models are simulated with MWRTM using the
    full \cite[Liebe1993] and \cite[Turner2016] absorption models based on
    high-resolution radiosonde data. Zero-mean random noise with a standard
    deviation of 0.5 K is added to all brightness temperature simulations used
    as input for retrievals when assessing the synthetic performance of optimal
    estimation and regression methods but not for regression training data
    (regularization is used instead). This is effective for the detection of
    overfitted regression models and is supposed to recreate the uncertainty of
    real measurements from a radiometer in order to increase the significance
    and representative value of synthetic retrieval performance assessments. As
    dicussed in section \in[ch:elevation_scanning], off-zenith data are only
    used from the four most opaque V band channels (54.94, 56.66, 57.30, 58.00
    GHz).

    The optimal estimation scheme uses MWRTM with fast absorption predictors
    for the forward and Jacobian calculations. The observation error covariance
    $\COVMATERR$ is determined from a comparison of MWRTM simulations with
    values from the Rosenkranz model from section \in[ch:model_comparison].
    Additionally, an uncorrelated instrument error of 0.5 K standard deviation
    is added to the observation error covariance consistent with the noise
    added to brightness temperature input data from MWRTM and the instrumental
    error assumed for HATPRO observations in section \in[ch:hatpro].

    The optimal estimation scheme is implemented with a Levenberg-Marquardt
    minimization scheme adapted from a form given by \cite[Rodgers2000]
    (equation 5.3.6)

    \startformula
        ((1 + \gamma) \COVMATA^{-1} + \ITER{\FWDJAC}{i}^\top \COVMATERR^{-1} \ITER{\FWDJAC}{i}) \,
        (\ITER{\MEANVEC}{i+1} - \ITER{\MEANVEC}{i})
            = \ITER{\FWDJAC}{i}^\top \COVMATERR^{-1} (\VECY - \FWD(\ITER{\MEANVEC}{i}) - \MEANVECERR)
                - \COVMATA^{-1} (\ITER{\MEANVEC}{i} - \MEANVECA) \EQSTOP
    \stopformula

    This equation is solved with a linear algebra routine for the difference
    $(\ITER{\MEANVEC}{i+1} - \ITER{\MEANVEC}{i})$ which is then used to
    determine $\ITER{\MEANVEC}{i+1}$, the new mean of the atmospheric state
    vector distribtion.

    $\gamma \ge 0$ is a parameter controlling the step size taken with each
    iteration. The inital value of $\gamma$ is set to 3000 and is then adapted
    after each iteration based on the change of the cost function
    \ineq{itercostfun}. If the cost function increases, $\gamma$ is increased
    by a factor of 5, if the cost function decreases $\gamma$ is halved. This
    strategy is not optimal but has been found to work well for retrievals here
    and was also used by \cite[Schneebeli2009].

    Convergence is determined from the change of the cost function
    \ineq{itercostfun}. If the cost function does not decrease by more than
    2 \% over 3 iterations, the iteration state with the smallest cost function
    value is returned as the retrieval result.  If no convergence is reached
    after 20 iterations, the iteration with the smallest cost function value is
    still returned but a quality flag is set indicating that the corresponding
    profile did not converge properly. The threshold of 2 \% was chosen
    subjectively based on experiences during experimentation.

    The basis functions chosen for all regression models are identity functions
    for the regressors and a constant function as shown in
    \ineq{regression_basis}. The regressors for temperature retrievals are
    brightness temperatures from V band channels, surface pressure and surface
    temperature. Specific water content retrievals use brightness temperatures
    from K band channels, surface pressure and surface specific humidity as
    regressors. Regularization was found to be important during training in
    order to avoid overfitting. The chosen regularization maximizes the
    performance of the regression models on the test data set.

\stopsection

