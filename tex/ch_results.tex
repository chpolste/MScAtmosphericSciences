In this chapter the accuracy of linear regression and optimal estimation
methods is assessed and the methods' behavior in selected case studies
investigated. Performance is determined based on simulated brightness
temperatures based on radiosonde data but also based on observations from
the HATPRO radiometer. The measures used to quantify accuracy are the mean
and standard deviation of retrieved profiles with respect to radiosonde data.
This measure gives an estimation of the average overall performance of a
retrieval method. Two case studies are also shown which give a more detailed
look into the behavior of the optimal estimation scheme and an experiment
with continuously propagated temperature retrievals is carried out for a day of
radiometer observations.

The focus of research into microwave radiometers in Innsbruck in the past has
been on the retrieval of vertical temperature profiles
\cite[authoryears][Massaro2013,Meyer2016]. Here, temperature profiles are also
treated with additional care while humidity and cloud liquid water content are
treated together as the specific water content of the atmosphere.

\startsection[title={Predictions without Radiometer Data}]

    Before the accuracy of retrieval methods is determined, the accuracy of
    the data these methods are based on should be assessed. Figure
    \in[fig:baseline] shows the standard deviation with respect to the
    radiosonde test data set of forecasts from the COSMO-7 numerical weather
    prediction model, of predicting the climatological mean temperature and
    water content in every situation and of a persistence forecast that
    predicts the atmospheric state at a given time as state the from the
    previous day. The COSMO-7 forecasts are available for two different lead
    times which are interpolations based on the  analysis field and a 6 hour
    forecast (+00+06) or a 24 and a 30 hour forecast (+24+30). 

    The climatological standard deviation indicates that the variability of
    temperature in the troposphere above Innsbruck is almost constant with
    height and declines at altitudes where the tropopause can be found. The
    variability of water content is highest near the surface and decreases
    linearly up to 5 km above which it appears to asymptotically approach zero.
    This behavior is mostly governed by the generally with height decreasing
    temperature in the tropopause which causes the saturation water vapor
    pressure to decrease exponentiall so that less and less water can be
    contained by the air at greater heights. The persistence forecast strategy
    has a fairly constant temperature accuracy of a little more than
    3 K throughout the troposphere It has better water content accuracy than
    the climatology forecast in the lowest 4 km of the atmosphere and performs
    similar aloft.

    The best accuracy exhibit COSMO-7 forecasts which are as accurate as
    a persistence forecast at the surface but have a standard deviation smaller
    than 1.5 K for all levels above 1 km height above the ground. The shorter
    +00+06 forecasts reach less than 1 K standard deviation error in the middle
    and upper troposphere while the +24+30 forecasts are not as accurate.
    Water contente forecasts from COSMO-7 do not vary in accuracy between the
    the two lead times evaluated here. They have a standard deviation of 1 g/kg
    at the surface which decreases approximately linearly with height. It is
    overall substantially smaller than the variability of the climatology.

    As expected, the COSMO-7 NWP model provides temperature forecasts of good
    accuracy at higher altitudes but with a comparatively poor performance in
    the lowest layers of the atmosphere. Because performance does not differ
    not substantially between the considered lead times, the +00+06 forecasts
    will be used as priors for all subsequent optimal estimation retrievals.

    \placefigure[top][fig:baseline]
        {Standard deviations of temperature (left) and total water content
        (right) profiles from COSMO-7 forecasts (black and grey), the
        climatological mean (blue) and persistence forecasts (green) with
        respect to the radiosonde test data set.  The +00+06 COSMO-7 profiles
        are interpolated to the reference radiosonde launch times based on
        forecasts with lead times of 0 (the analysis fields) and 6 hours while
        the +24+36 profiles are interpolated based on profiles with lead times
        of 24 and 36 hours. The evaluation of the persistence forecasts is
        performed on a reduced data set of 212 profiles due to missing days in
        the test data set.
        }
        {\externalfigure[retrieval_baseline][width=\textwidth]}

\stopsection


\startsection[title={Statistical Retrieval Performance}]

    The statistical performance of different retrieval schemes is first
    assessed based on simulations of brightness temperatures by MWRTM and then
    investigated with data from an actual radiometer.

\startsubsection[title={Linear Regression Retrievals}]

    Figure \in[fig:reg_default] shows the statistical evaluation of linear
    regression models used to retrieve temperature and specific water content.
    These models use surface pressure and a surface value of the retrieved
    quantity as additional regressors which have are taken from the lowest
    level of the retrieval grid for the purposes of this synthetic performance
    evaluation.

    \placefigure[top][fig:reg_default]
        {Performance evaluation of linear regression retrievals of temperature
        (left) and specific water content (right) in terms of bias (dashed) and
        standard deviation (solid). Temperature retrievals are based on
        MWRTM-simulated brightness temperatures in the V band at zenith (black)
        or zenith with additional elevation scan information from 4 channels
        (green).  Water content retrievals are based on MWRTM-simulated
        brightness temperatures in the K band at zenith. The black model uses
        all training and test data while the blue model is only trained with
        and tested on clear sky profiles.  Surface values of pressure and
        temperature/humidity are used as additional regressors. The lower
        panels are enlarged views of the shaded regions in the upper panels.
        }
        {\externalfigure[retrieval_regression][width=\textwidth]}

    For temperature retrievals accuracy is generally decreasing with height
    starting from an accuracy of less than 0.4 K at the surface and reaching
    3 K at approximately 9 km height above ground. The use of elevation scan
    information improves the retrieval accuracy in the lowest 2 km and reduces
    the bias in the middle troposphere. Accuracy of the model using elevation
    scan data is worse between 2 and 5 km. This issue could be resolved by
    gradually removing elevation scan brightness temperatures as regressors
    with height as the information content of the boundary layer scans is
    restricted to the lower atmosphere.

    The regression retrievals of temperature have no significant bias in the
    lowest 1 km of the atmosphere where their standard deviations are less than
    1 K. Accuracy decreases to 2 K at 4 km height and increases further with
    height troughout the rest of the troposphere. The accuracies found by
    \cite[Massaro2015,Meyer2016] for linear regression models used with
    Innsbruck data were of similar magnitude although the decrease of accuracy
    of their regression models was not as high as the one observed here.
    Possible reasons for this discrepancy are the use of regression models with
    quadratic terms by these authors and different assumptions about noise
    added to the test data.

    The regression retrievals have higher accuracy for water content in the
    lowest 2 km than COSMO-7 forecasts but do not perform better than the NWP
    model at higher levels. If a regression model is trained with clear sky
    cases only, it shows better accuracy in the lowest 2 km of the atmosphere
    than an all-sky trained model when evaluated on clear sky cases only but
    the specialized model has an increased bias. Specializing on cloudyness for
    temperature models caused no change of accuracy of bias of the retrieved
    profiles.

\stopsubsection

\startsubsection[title={Optimal Estimation Retrievals with COSMO-7 Prior}]

    Figure \in[fig:ret_optest] shows the statistical evaluation of retrievals
    by an optimal estimation scheme using COSMO-7 forecasts as a prior for
    the atmospheric state variables. Retrievals with all brightness temperature
    information were found to converge 96 \% of the time with an average of
    11.9 iterations. Retrieval calculations took between 8 and 15 seconds on a
    modern multicore processor showing that MWRTM simulations are fast enough
    for operational retrieval applications.

    \placefigure[top][fig:ret_optest]
        {Performance evaluation of optimal estimation retrievals of temperature
        (left) and specific water content (right) in terms of bias (dashed) and 
        standard deviation (solid). An evaluation of COSMO-7 profiles
        interpolated from analysis fields and 6 h forecasts is shown for
        comparison in black. The optimal estimation retrievals are based on
        all available brightness temperature information (simulated by MWRTM)
        (blue) or only zenith data (gray, only relevant for temperature
        retrievals). The lower panels are enlarged views of the shaded regions
        in the upper panels.
        }
        {\externalfigure[retrieval_optest][width=\textwidth]}

    The accuracy of temperature profiles obtained from this method is less than
    1 K throughout the troposphere with the exception of a few layers at 3 km
    height above ground. Comparison with the COSMO-7 performance shows that
    the optimal estimation scheme relies on the temperature forcast of the NWP
    model forecast for all levels above 2.5 km. The radiometer information
    however improve accuracy compared to the model output substantially in the
    lower levels. The use of zenith observations alone reduces the temperature
    accuracy by approximately 0.2 K in the lowest 2 km which was also found
    for the linear regression model. Performance of the retrieval scheme is
    worst in heights between 1.5 and 3 km where the radiometer information
    content decreases and the NWP model forecasts are not as accurate as in
    the upper troposphere. The temperature bias of the optimal estimation
    retrievals is very low. This can partially be attributed to the
    bias-corrected COSMO-7 priors.

    The radiometer observations improve the water content accuracy of COSMO-7
    forecasts by more than 0.2 g/kg in the lowest 3 km of the atmosphere when
    incorporated into the prior information by the optimal estimation technique
    and have a positive effect up to 6 km height above ground. Bias values are
    however higher below 2 km compared to the bias of the COSMO-7 forecasts.
    Note that the COSMO-7 bias of water content was corrected in logarithmic
    space during the processing of the model output and is therefore not zero
    in a non-logarithmic atmospheric state space.

\stopsubsection

\startsubsection[title={Combined Approaches}]

    It is of interest to see if a combination of regression and optimal
    estimation methods exists that improves upon the retrieval result of
    optimal estimation alone. Two schemes were set up to investigate this,
    one using the atmospheric state retrieved from a linear regression model
    as the first guess of the iteration procedure and one using a prior
    distribution constructed from the error characteristics of a linear
    regression model using COSMO-7 forecasts as first guesses of the iteration.

    As seen in Figure \in[fig:ret_combined] on the black and gray curves
    temperature accuracy is not improved in the lowest 1.5 km by either
    combined approach. The scheme using a prior distribution based on
    regression retrievals has no better accuracy than regression alone at
    levels higher than 1.5 K. Water content retrievals with regression-derived
    first guesses show no significant improvements of statistical accuracy too.
    The retrieval scheme with the regression prior inherits the smaller
    bias of water content in the lower troposphere.

    Also shown in Figure \in[fig:ret_combined] is the performance evaluation of
    a linear regression method using COSMO-7 forecast data from the middle and
    upper troposphere for temperature and water content data in the lower
    atmosphere. The temperature forecasts make a big difference in terms of
    accuracy and bias for the temperature retrievals. The regression model
    is however not able to use the full potential of the COSMO-7 data at
    height and still has less statistical accuracy than the optimal estimation
    scheme or even the COSMO-7 forecasts by themselves. Reducing the number of
    brightness temperature regressors at higher levels would likely amend this
    issue.

    Because the COSMO-7 forecasts are not more accurate in the lower atmosphere
    than linear regression retrievals of water content alone no improvement
    is seen when incorporating such forecasts into a linear regression model.

    \placefigure[top][fig:ret_combined]
        {Performance evaluation of combined linear regression/optimal
        estimation retrievals of temperature (left) and specific water content
        (right) using either prior distributions from COSMO-7 forecasts and
        a first guess determined from a regression model (black) or a prior
        constructed from a set of linear regression model retrievals and
        a COSMO-7 forecast as the first guess (grey). The green model is
        a linear regression model using COSMO-7 forecasted temperatures from
        the 2104, 2694, 3448, 4415, 5651, 7235, 9262 and 11857 m levels or
        specific water content from the 612, 783, 1003, 1284, 1644 levels as
        additional regressors. Because the regression model can only be
        trained on the data set, 4-fold cross validation was used for
        the performance assessment.  The lower panels are enlarged views of the
        shaded regions in the upper panels.
        }
        {\externalfigure[retrieval_combined][width=\textwidth]}

\stopsubsection

\startsubsection[title={Retrievals with Actual Radiometer Data}]

    Bias correction of HATPRO data reduce bias (but data sample very small).
    Bias has no influence on standard deviation. Performance for the test
    profiles better at height than with synthetic data for temperature worse
    in lower 2 km for humidity (representativity of radiosonde for humidity?).

    Nocturnal boundary seems to differ substantially between Innsbruck airport
    where the radiosondes launch and Innsbruch university where the radiometer
    is. Substantially increased standard deviation in lowest 500 m. Could be
    due to slightly different altitudes of sites and boundary layer
    inhomogeneity.

    \placefigure[top][fig:ret_hatpro]
        {Performance evaluation of linear regression (green) and optimal
        estimation (blue, grey) retrievals of temperature and specific water
        content based on brightness temperature measurements from a radiometer
        (HATPRO). The test data set are 26 radiosonde ascents with simultaneous
        radiometer observations. COSMO-7 forecasts are used as the prior and
        first guess of the optimal estimation retrievals. The regression models
        use brightness temperatures and surface pressure and
        temperature/humidity observations as regressors. The bias between MWRTM
        and HATPRO determined in section \in[ch:hatpro] has been corrected for
        the green and blue models but not for the grey.  The lower panels are
        enlarged views of the shaded regions in the upper panels.
        }
        {\externalfigure[retrieval_hatpro][width=\textwidth]}

\stopsubsection

\stopsection


\startsection[title={Selected Case Studies}]

Special interest in inversions and influence of prior. Selection due to
availability of actual radiometer data and atmospheric features.

\startsubsection[title={Ground-based Temperature Inversion}]

    2015-10-28 02:15:05
    Some cirrus clouds, no liquid clouds in lower atmosphere.

    \placefigure[top][fig:ret_case1]
        {Retrievals of temperature (left) and specific water content (right)
        by a linear regression model (green) and the optimal estimation
        technique (blue) compared to the radiosounding (black) from which
        brightness temperatures were simulated by MWRTM. The radiosonde was
        launched on 2015-10-28 at 02:15:05 UTC from Innsbruck airport. The
        blue shaded region shows the ± one standard deviation uncertainty
        bounds obtained from sampling the posterior distribution of the
        optimal estimation retrieval.  The lower panels are enlarged views of
        the shaded regions in the upper panels.
        }
        {\externalfigure[retrieval_case1][width=\textwidth]}

    \placefigure[top][fig:ret_iteration]
        {Visualization of intermediate profiles of temperature from the
        iteration procedure of the optimal estimation retrieval from Figure
        \in[fig:ret_case1]. The line labeled COSMO-7 shows the first guess
        and the blue line the profile at convergence. The profile at iteration
        6 is shown in both plots for comparison purposes.
        }
        {\externalfigure[retrieval_iteration][width=\textwidth]}


\stopsubsection

\startsubsection[title={Elevated Temperature Inversion}]

    2015-09-11 03:48:00
    Elevated temperature inversion, partial low status deck forming but never
    closed.

    \placefigure[top][fig:ret_elevated]
        {Optimal estimation retrievals of temperature for an elevated boundary
        layer inversion case from 2015-09-11 at 03:48:00 UTC (time of
        radiosonde launch). The right panel shows the temperature profile
        (blue) retrieved default the model setup with a prior and first guess
        from a COSMO-7 forecast (grey) as well as a retrieval from a linear
        regression model (green). The middle panel shows an optimal estimation
        retrieval (blue) with the same COSMO-7 prior (now light grey) but
        a first guess that has been modified to include an elevated temperature
        inversion of 1 K strength (dark grey). The right panel shows an optimal
        estimation retrieval (blue) with a prior that has its covariance from
        the COSMO-7 forecast but whose mean is the first guess from the middle
        panel (dark grey) which is also the first guess here.
        }
        {\externalfigure[retrieval_elevated][width=\textwidth]}


\stopsubsection

\startsubsection[title={Boundary Layer Evolution}]

    Start from ground based inversion case, use radiosonde as first profile
    with cosmo-7 uncertainty (most likely to fit and no better alternative) and
    continuously move forward in time with each previous profile as the
    starting point of the next.
    Then verify profile at the end.

    \placefigure[top][fig:ret_continuous]
        {Visualization of the temperature evolution as retrieved from a linear
        regression model (left) and the optimal estimation technique (right)
        based on actual radiometer observations in the time span from
        2015-10-28 02:15 UTC to 2015-10-29 02:15 UTC. Profiles are retrieved
        with a temporal resolution of 10 minutes. The prior covariance of the
        optimal estimation retrievals is that of COSMO-7 forecasts and the
        prior means are the retrieved atmospheric states from the previous
        timestep. The initial profile is a radiosonde profile from Innsbruck
        airport.
        }
        {\externalfigure[retrieval_continuous][width=\textwidth]}


\stopsubsection

\stopsection

