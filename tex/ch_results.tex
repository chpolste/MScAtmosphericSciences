In this chapter the accuracy of linear regression and optimal estimation
methods is assessed and the methods' behavior in selected case studies
investigated. Performance is determined based on simulated brightness
temperatures based on radiosonde data but also based on observations from
the HATPRO radiometer. The measures used to quantify accuracy are the mean
and standard deviation of retrieved profiles with respect to radiosonde data.
This measure gives an estimation of the average overall performance of a
retrieval method. Two case studies are also shown which give a more detailed
look into the behavior of the optimal estimation scheme and an experiment
with continuously propagated temperature retrievals is carried out for a day of
radiometer observations.

The focus of research into microwave radiometers in Innsbruck in the past has
been on the retrieval of vertical temperature profiles
\cite[authoryears][Massaro2013,Meyer2016]. Here, temperature profiles are also
treated with additional care while humidity and cloud liquid water content are
treated together as the specific water content of the atmosphere.

\startsection[title={Predictions without Radiometer Data}]

    Before the accuracy of retrieval methods is determined, the accuracy of
    the data these methods are based on should be assessed. Figure
    \in[fig:baseline] shows the standard deviation with respect to the
    radiosonde test data set of forecasts from the COSMO-7 numerical weather
    prediction model, of predicting the climatological mean temperature and
    water content in every situation and of a persistence forecast that
    predicts the atmospheric state at a given time as state the from the
    previous day. The COSMO-7 forecasts are available for two different lead
    times which are interpolations based on the  analysis field and a 6 hour
    forecast (+00+06) or a 24 and a 30 hour forecast (+24+30). 

    The climatological standard deviation indicates that the variability of
    temperature in the troposphere above Innsbruck is almost constant with
    height and declines at altitudes where the tropopause can be found. The
    variability of water content is highest near the surface and decreases
    linearly up to 5 km above which it appears to asymptotically approach zero.
    This behavior is mostly governed by the generally with height decreasing
    temperature in the tropopause which causes the saturation water vapor
    pressure to decrease exponentially so that less and less water can be
    contained by the air at greater heights. The persistence forecast strategy
    has a fairly constant temperature accuracy of a little more than
    3 K throughout the troposphere It has better water content accuracy than
    the climatology forecast in the lowest 4 km of the atmosphere and performs
    similar aloft.

    The best accuracy exhibit COSMO-7 forecasts which are as accurate as
    a persistence forecast at the surface but have a standard deviation smaller
    than 1.5 K for all levels above 1 km height above the ground. The shorter
    +00+06 forecasts reach less than 1 K standard deviation error in the middle
    and upper troposphere while the +24+30 forecasts are not as accurate.
    Water content forecasts from COSMO-7 do not vary in accuracy between the
    the two lead times evaluated here. They have a standard deviation of 1 g/kg
    at the surface which decreases approximately linearly with height. It is
    overall substantially smaller than the variability of the climatology.

    As expected, the COSMO-7 NWP model provides temperature forecasts of good
    accuracy at higher altitudes but with a comparatively poor performance in
    the lowest layers of the atmosphere. Because performance does not differ
    not substantially between the considered lead times, the +00+06 forecasts
    will be used as priors for all subsequent optimal estimation retrievals.

    \placefigure[top][fig:baseline]
        {Standard deviations of temperature (left) and total water content
        (right) profiles from COSMO-7 forecasts (black and grey), the
        climatological mean (blue) and persistence forecasts (green) with
        respect to the radiosonde test data set.  The +00+06 COSMO-7 profiles
        are interpolated to the reference radiosonde launch times based on
        forecasts with lead times of 0 (the analysis fields) and 6 hours while
        the +24+36 profiles are interpolated based on profiles with lead times
        of 24 and 36 hours. The evaluation of the persistence forecasts is
        performed on a reduced data set of 212 profiles due to missing days in
        the test data set.
        }
        {\externalfigure[retrieval_baseline][width=\textwidth]}

\stopsection


\startsection[title={Statistical Retrieval Performance}]

    The statistical performance of different retrieval schemes is first
    assessed based on simulations of brightness temperatures by MWRTM and then
    investigated with data from an actual radiometer.

\startsubsection[title={Linear Regression Retrievals}]

    Figure \in[fig:reg_default] shows the statistical evaluation of linear
    regression models used to retrieve temperature and specific water content.
    These models use surface pressure and a surface value of the retrieved
    quantity as additional regressors which have are taken from the lowest
    level of the retrieval grid for the purposes of this synthetic performance
    evaluation.

    \placefigure[top][fig:reg_default]
        {Performance evaluation of linear regression retrievals of temperature
        (left) and specific water content (right) in terms of bias (dashed) and
        standard deviation (solid). Temperature retrievals are based on
        MWRTM-simulated brightness temperatures in the V band at zenith (black)
        or zenith with additional elevation scan information from 4 channels
        (green).  Water content retrievals are based on MWRTM-simulated
        brightness temperatures in the K band at zenith. The black model uses
        all training and test data while the blue model is only trained with
        and tested on clear sky profiles.  Surface values of pressure and
        temperature/humidity are used as additional regressors. The lower
        panels are enlarged views of the shaded regions in the upper panels.
        }
        {\externalfigure[retrieval_regression][width=\textwidth]}

    For temperature retrievals accuracy is generally decreasing with height
    starting from an accuracy of less than 0.4 K at the surface and reaching
    3 K at approximately 9 km height above ground. The use of elevation scan
    information improves the retrieval accuracy in the lowest 2 km and reduces
    the bias in the middle troposphere. Accuracy of the model using elevation
    scan data is worse between 2 and 5 km. This issue could be resolved by
    gradually removing elevation scan brightness temperatures as regressors
    with height as the information content of the boundary layer scans is
    restricted to the lower atmosphere.

    The regression retrievals of temperature have no significant bias in the
    lowest 1 km of the atmosphere where their standard deviations are less than
    1 K. Accuracy decreases to 2 K at 4 km height and increases further with
    height throughout the rest of the troposphere. The accuracies found by
    \cite[Massaro2015,Meyer2016] for linear regression models used with
    Innsbruck data were of similar magnitude although the decrease of accuracy
    of their regression models was not as high as the one observed here.
    Possible reasons for this discrepancy are the use of regression models with
    quadratic terms by these authors and different assumptions about noise
    added to the test data.

    The regression retrievals have higher accuracy for water content in the
    lowest 2 km than COSMO-7 forecasts but do not perform better than the NWP
    model at higher levels. If a regression model is trained with clear sky
    cases only, it shows better accuracy in the lowest 2 km of the atmosphere
    than an all-sky trained model when evaluated on clear sky cases only but
    the specialized model has an increased bias. Specializing on cloudiness for
    temperature models caused no change of accuracy of bias of the retrieved
    profiles.

\stopsubsection

\startsubsection[title={Optimal Estimation Retrievals with COSMO-7 Prior}]

    Figure \in[fig:ret_optest] shows the statistical evaluation of retrievals
    by an optimal estimation scheme using COSMO-7 forecasts as a prior for
    the atmospheric state variables. Retrievals with all brightness temperature
    information were found to converge 96 \% of the time with an average of
    11.9 iterations. Retrieval calculations took between 8 and 15 seconds on a
    modern multi core processor showing that MWRTM simulations are fast enough
    for operational retrieval applications.

    \placefigure[top][fig:ret_optest]
        {Performance evaluation of optimal estimation retrievals of temperature
        (left) and specific water content (right) in terms of bias (dashed) and 
        standard deviation (solid). An evaluation of COSMO-7 profiles
        interpolated from analysis fields and 6 h forecasts is shown for
        comparison in black. The optimal estimation retrievals are based on
        all available brightness temperature information (simulated by MWRTM)
        (blue) or only zenith data (gray, only relevant for temperature
        retrievals). The lower panels are enlarged views of the shaded regions
        in the upper panels.
        }
        {\externalfigure[retrieval_optest][width=\textwidth]}

    The accuracy of temperature profiles obtained from this method is less than
    1 K throughout the troposphere with the exception of a few layers at 3 km
    height above ground. Comparison with the COSMO-7 performance shows that
    the optimal estimation scheme relies on the temperature forecast of the NWP
    model forecast for all levels above 2.5 km. The radiometer information
    however improve accuracy compared to the model output substantially in the
    lower levels. The use of zenith observations alone reduces the temperature
    accuracy by approximately 0.2 K in the lowest 2 km which was also found
    for the linear regression model. Performance of the retrieval scheme is
    worst in heights between 1.5 and 3 km where the radiometer information
    content decreases and the NWP model forecasts are not as accurate as in
    the upper troposphere. The temperature bias of the optimal estimation
    retrievals is very low. This can partially be attributed to the
    bias-corrected COSMO-7 priors.

    The temperature retrieval accuracy can be compared to studies by
    \cite[Cimini2011,Martinet2015] who used optimal estimation techniques
    together with analyses and forecasts from a regional NWP model
    respectively. An accuracy of 1 K and better was found in both studies for
    the troposphere. The highest standard deviation is located around 2 km
    height in the evaluation of \cite[Cimini2011] and around 1.7 km height
    in the study of \cite[Martinet2015] similar to the results shown here. 

    The radiometer observations improve the water content accuracy of COSMO-7
    forecasts by more than 0.2 g/kg in the lowest 3 km of the atmosphere when
    incorporated into the prior information by the optimal estimation technique
    and have a positive effect up to 6 km height above ground. Bias values of
    water content are however higher below 2 km compared to the bias of the
    COSMO-7 forecasts. Note that the COSMO-7 bias of water content was
    corrected in logarithmic space during the processing of the model output
    and is therefore not zero in a non-logarithmic atmospheric state space.

\stopsubsection

\startsubsection[title={Combined Approaches}]

    It is of interest to see if a combination of regression and optimal
    estimation methods exists that improves upon the retrieval result of
    optimal estimation alone. Two schemes were set up to investigate this,
    one using the atmospheric state retrieved from a linear regression model
    as the first guess of the iteration procedure and one using a prior
    distribution constructed from the error characteristics of a linear
    regression model using COSMO-7 forecasts as first guesses of the iteration.

    As seen in Figure \in[fig:ret_combined] on the black and gray curves
    temperature accuracy is not improved in the lowest 1.5 km by either
    combined approach. The scheme using a prior distribution based on
    regression retrievals has no better accuracy than regression alone at
    levels higher than 1.5 K. Water content retrievals with regression-derived
    first guesses show no significant improvements of statistical accuracy too.
    The retrieval scheme with the regression prior inherits the smaller
    bias of water content in the lower troposphere.

    Also shown in Figure \in[fig:ret_combined] is the performance evaluation of
    a linear regression method using COSMO-7 forecast data from the middle and
    upper troposphere for temperature and water content data in the lower
    atmosphere. The temperature forecasts make a big difference in terms of
    accuracy and bias for the temperature retrievals. The regression model
    is however not able to use the full potential of the COSMO-7 data at
    height and still has less statistical accuracy than the optimal estimation
    scheme or even the COSMO-7 forecasts by themselves. Reducing the number of
    brightness temperature regressors at higher levels would likely amend this
    issue.

    Because the COSMO-7 forecasts are not more accurate in the lower atmosphere
    than linear regression retrievals of water content alone no improvement
    is seen when incorporating such forecasts into a linear regression model.

    \placefigure[top][fig:ret_combined]
        {Performance evaluation of combined linear regression/optimal
        estimation retrievals of temperature (left) and specific water content
        (right) using either prior distributions from COSMO-7 forecasts and
        a first guess determined from a regression model (black) or a prior
        constructed from a set of linear regression model retrievals and
        a COSMO-7 forecast as the first guess (gray). The green model is
        a linear regression model using COSMO-7 forecasted temperatures from
        the 2104, 2694, 3448, 4415, 5651, 7235, 9262 and 11857 m levels or
        specific water content from the 612, 783, 1003, 1284, 1644 levels as
        additional regressors. Because the regression model can only be
        trained on the data set, 4-fold cross validation was used for
        the performance assessment.  The lower panels are enlarged views of the
        shaded regions in the upper panels.
        }
        {\externalfigure[retrieval_combined][width=\textwidth]}

\stopsubsection

\startsubsection[title={Retrievals with Real Radiometer Data},reference=ch:hatproeval]

    Retrieval results from elevation scan brightness temperature measurements
    by the HATPRO radiometer in Innsbruck are evaluated in Figure
    \in[fig:ret_hatpro]. Temperature profiles obtained from optimal estimation
    have similar accuracy to the one determined from synthetic retrievals, with
    highest standard deviations at heights around 2 km and overall accuracy
    better than 1 K almost everywhere. A notable difference from the synthetic
    retrievals is the standard deviation maximum in the 300 m above the ground.
    A similar spike in standard deviation is found for the water content at
    heights up to 600 m. Considering the differing locations and altitudes of
    the radiometer site and the radiosonde launch site, it seems likely that
    this is a representativity issue. The balloon probably experiences
    a different nocturnal boundary layer profiles near the airport than the
    radiometer which is located higher and in an urban area.

    \placefigure[top][fig:ret_hatpro]
        {Performance evaluation of linear regression (green) and optimal
        estimation (blue, gray) retrievals of temperature and specific water
        content based on brightness temperature measurements from a radiometer
        (HATPRO). The test data set are 26 radiosonde ascents with simultaneous
        radiometer observations. COSMO-7 forecasts are used as the prior and
        first guess of the optimal estimation retrievals. The regression models
        use brightness temperatures and surface pressure and
        temperature/humidity observations as regressors. The bias between MWRTM
        and HATPRO determined in section \in[ch:hatpro] has been corrected for
        the green and blue models but not for the gray.  The lower panels are
        enlarged views of the shaded regions in the upper panels.
        }
        {\externalfigure[retrieval_hatpro][width=\textwidth]}

    Bias values of the statistical temperature and water content comparison
    are high, particularly those of temperature in the lowest 2 km. The gray
    curve shows that the bias correction determined in section \in[ch:hatpro]
    affects temperature retrievals strongest in the lowest 500 m of the
    atmosphere and reduces the bias of water content retrievals by up to 0.5
    g/kg in the lower troposphere. A very similar behavior is found for the
    performance of a linear regression model (not shown): if the brightness
    temperatures from HATPRO are not treated with a bias correction, water
    content retrievals are strongly biased towards higher amounts.

    It is interesting to see that the temperature accuracy in the lowest 400 m
    of the atmosphere is substantially worse when using the 
    Nocturnal boundary seems to differ substantially between Innsbruck airport
    where the radiosondes launch and Innsbruck university where the radiometer
    is. Substantially increased standard deviation in lowest 500 m. Could be
    due to slightly different altitudes of sites and boundary layer
    inhomogeneity.

    Definitive conclusions from these results cannot be drawn from these data
    due to the small sample size. It indicates however that the retrieval
    methods and particularly the radiative transfer model MWRTM are useful
    when applied to actual radiometer measurements. For temperature retrievals
    comparable accuracy can be achieved to the synthetic statistical
    performance. The accuracy of water content retrievals is not as good with
    standard deviation values of 1 g/kg in the lower troposphere.

\stopsubsection

\stopsection


\startsection[title={Selected Case Studies}]

Now that the general reasonability of retrieval results from the optimal
estimation method has been shown, its behavior is examined for two selected
cases: a ground-based and an elevated temperature inversion. These test
scenarios are commonly investigated \cite[alternative=authoryears,left={(e.g.
by }][Hewison2004,Crewell2007] and have been at the focus of research in
Innsbruck before \cite[authoryears][Massaro2013,Meyer2016]. The presentation
of results is concluded with a visualization of a boundary layer evolution over
24 h as retrieved by a linear regression and optimal estimation method from
HATPRO data.

\startsubsection[title={Ground-based Temperature Inversion},reference={ch:gbinv}]

    A ground-based temperature inversion of 5 K strength was measured during a
    radiosonde ascent started on the 2015-10-28 at 02:15:05 UTC. Aside from
    some cirrus no liquid clouds existed that night at Innsbruck\footnote{These
    information are based on web cam pictures found at
    \hyphenatedurl{http://www.foto-webcam.eu/webcam/innsbruck/}}.

    \placefigure[top][fig:ret_case1]
        {Retrievals of temperature (left) and specific water content (right)
        by a linear regression model (green) and the optimal estimation
        technique (blue) compared to the radiosounding (black) from which
        brightness temperatures were simulated by MWRTM. The radiosonde was
        launched on 2015-10-28 at 02:15:05 UTC from Innsbruck airport. The
        blue shaded region shows the ± one standard deviation uncertainty
        bounds obtained from sampling the posterior distribution of the
        optimal estimation retrieval.  The lower panels are enlarged views of
        the shaded regions in the upper panels.
        }
        {\externalfigure[retrieval_case1][width=\textwidth]}

    Figure \in[fig:ret_case1] shows temperature and water content retrievals
    from a linear regression model and an optimal estimation scheme based on
    a COSMO-7 prior. Both methods are able to retrieve a temperature inversion
    of appropriate strength although the regression model places the maximum
    about 100 m too high. The optimal estimation scheme is better able to
    retrieve the temperature curve above the maximum and starts to deviate from
    the radiosonde profile at 1.5 km height above ground while the regression
    model is consistently about 1 K warmer than the radiosounding up to the
    weak second inversion at 2 km which both methods fail to capture. It is
    however included in the one standard deviation uncertainty range of the
    optimal estimation retrieval while no uncertainty estimate is available
    for the regression model.

    Figure \in[fig:ret_iteration] shows temperature profiles from intermediate
    steps of the optimal estimation iterative procedure. The mean of the
    COSMO-7 prior from which the iteration starts has a ground-based
    temperature inversion which is rather weak and too close to the ground
    compared to radiosonde profile. In the first 6 iterations the optimal
    estimation retrieval moves this strengthens this inversion by 2 K and
    lifts its maximum to the appropriate height. At this stage temperatures
    are significantly lower than measured by the radiosonde. The magnitude
    of the inversion is adjusted with the next 7 iterations that add another
    2 K to the maximum and adjust the temperature above the inversion to
    values similar to the radiosonde profile. Convergence is then reached at
    iteration 13.

    The water content profile with a moist layer from 2 to 4 km is not captured
    by any of the methods. Both models resort to a curve that has too high
    values of humidity in the lower levels and too low values in the higher
    levels. The moist layer is covered by the uncertainty range of the optimal
    estimation scheme but the dryer layer below falls outside the one standard
    deviation range.

    \placefigure[top][fig:ret_iteration]
        {Visualization of intermediate profiles of temperature from the
        iteration procedure of the optimal estimation retrieval from Figure
        \in[fig:ret_case1]. The line labeled COSMO-7 shows the first guess
        and the blue line the profile at convergence. The profile at iteration
        6 is shown in both plots for comparison purposes.
        }
        {\externalfigure[retrieval_iteration][width=\textwidth]}

\stopsubsection

\startsubsection[title={Elevated Temperature Inversion}]

    The second case of interest is an elevated temperature inversion of 2 K
    magnitude measured by a radiosonde launched on 2015-09-11 at 03:48:00 UTC
    at Innsbruck airport. Web cam pictures of the night show low-level clouds
    forming and dissolving at the height of this inversion but a full stratus
    deck is never built up. The radiosonde profile was classified as cloudy
    based on its measured humidity values at the inversion height.

    \placefigure[top][fig:ret_elevated]
        {Optimal estimation retrievals of temperature for an elevated boundary
        layer inversion case from 2015-09-11 at 03:48:00 UTC (time of
        radiosonde launch). The right panel shows the temperature profile
        (blue) retrieved default the model setup with a prior and first guess
        from a COSMO-7 forecast (gray) as well as a retrieval from a linear
        regression model (green). The middle panel shows an optimal estimation
        retrieval (blue) with the same COSMO-7 prior (now light gray) but
        a first guess that has been modified to include an elevated temperature
        inversion of 1 K strength (dark gray). The right panel shows an optimal
        estimation retrieval (blue) with a prior that has its covariance from
        the COSMO-7 forecast but whose mean is the first guess from the middle
        panel (dark gray) which is also the first guess here.
        }
        {\externalfigure[retrieval_elevated][width=\textwidth]}

    The left panel of Figure \in[fig:ret_elevated] shows that the COSMO-7 prior
    mean does not have an elevated inversion but much warmer temperatures in
    the lowest 500 m above the ground. The optimal estimation retrieval is able
    to derive an elevated inversion at the appropriate height from this first
    guess but fails to adequately capture its magnitude. A regression retrieval
    also indicates an elevated inversion with its smoothed out S-shape but
    the magnitude is even weaker than that of the optimal estimation profile.

    If it were known that an elevated temperature inversion exists, e.g.
    by seeing that a partial low stratus deck has formed on web cam pictures,
    the first guess for the optimal estimation could be modified to include
    such an inversion. This has been done in the center panel of Figure
    \in[fig:ret_elevated]: the COSMO-7 prior mean was disturbed by + 0.5 K
    in a few layers above 400 m and by - 0.5 K in the layers below 400 m.
    This profile has been used as the first guess for the shown optimal
    estimation retrieval. No improvement in the depiction of the inversion
    results from this change. If however the prior mean is changed together
    with the first guess, the elevated inversion is captured much better by
    the optimal estimation method (right panel). This shows that the prior
    distribution is important even at low levels where the retrieved profile
    is governed mostly by the radiometer observations. It also shows that
    there is much potential for retrievals of elevated temperature inversions
    if an appropriate prior can be found. The absolute values of the prior do
    not seem to matter as much as the shape of its mean which should fit that
    of the actual atmospheric state for best results.

\stopsubsection

\startsubsection[title={Boundary Layer Evolution}]

    Starting from the ground based inversion case discussed in section
    \in[ch_gbinv], the temperature evolution in the boundary layer has been
    retrieved by an optimal estimation scheme based on HATPRO elevation scan
    observations of 10 minute resolution. Instead of using a COSMO-7 prior
    and first guess, the scheme starts with the radiosonde profile from
    2015-10-28 02:15:05 UTC and retrieves temperature from each HATPRO
    observation starting from the atmospheric state retrieved from the previous
    measurement. Because a covariance for the prior is needed and cannot be
    sampled from any available data, the COSMO-7 covariance is chosen as the
    best available alternative. Such a continuous retrieval scheme could be
    used operationally when no NWP forecasts are available. It should however
    only be used for profiling of the boundary layer because the information
    content of the radiometer is small at mid- and upper-tropospheric levels
    and no reasonable evolution of the profile will take place there.

    The retrieved temperature field is visualized in the right panel of Figure
    \in[fig:ret_continuous]. It can be compared to the temperature field
    retrieved by a linear regression model shown in the left panel. Overall,
    the fields of both retrieval schemes are similar. The nocturnal
    ground-based inversion is dissolved around 9:00 UTC in both fields and
    a new and inversion forms after 19:00 UTC. The radiosonde profile from
    2015-10-29 02:15 UTC, i.e. the end of the shown timespan, has
    a ground-based temperature inversion of 5 K strength with a maximum of 285
    K at approximately 400 m height above ground (not shown). Both retrieval
    schemes do not show such properties but the representativity issue
    found in section \in[ch:hatproeval] has to be kept in mind and this
    discrepancy might not be a failure of the retrieval methods.

    \placefigure[top][fig:ret_continuous]
        {Visualization of the temperature evolution as retrieved from a linear
        regression model (left) and the optimal estimation technique (right)
        based on actual radiometer observations in the time span from
        2015-10-28 02:15 UTC to 2015-10-29 02:15 UTC. Profiles are retrieved
        with a temporal resolution of 10 minutes. The prior covariance of the
        optimal estimation retrievals is that of COSMO-7 forecasts and the
        prior means are the retrieved atmospheric states from the previous
        timestep. The initial profile is a radiosonde profile from Innsbruck
        airport.
        }
        {\externalfigure[retrieval_continuous][width=\textwidth]}

    Because no temperature soundings exist for the day, the temperature
    evolution has to be judged qualitatively and cannot be verified
    quantitatively. The temperature maximum at the ground around 16:00 local
    time seems reasonable and both retrieval methods agree on the timing and
    value of this maximum. Based on the height of the 284 K isoline, the
    boundary layer grows by approximately 200 m in both fields although its top
    is more stable in the optimal estimation retrievals. This can be seen from
    the distance between the 284 and 281 K isolines.

\stopsubsection

\stopsection

