Main measure for general performance assessment: bias and root mean square
error (rmse).

\startsection[title={Predictions without Radiometer Data}]

    Establish baseline performance with climatology, persistence and COSMO-7.

    \placefigure[top][fig:baseline]
        {Root mean square errors with respect to the radiosonde test data set
        of COSMO-7 forecasts (black and grey), climatological mean (blue) and
        persistence forecast (green). The +00+06 COSMO-7 profiles are
        interpolated to the reference radiosonde launch times based on
        forecasts with lead times of 0 and 6 hours while the +24+36 profiles 
        are interpolated based on profiles with lead times of 24 and 36 hours.
        The evaluation of the persistence forecasts is performed on a reduced
        data set due to data gaps in the test set.
        }
        {\externalfigure[retrieval_baseline][width=\textwidth]}

\stopsection

\startsection[title={Synthetic Retrieval Performance}]

    Cloudy vs. clear sky
    Zenith vs. full elevations

\startsubsection[title={Regression Retrievals}]


\stopsubsection

\startsubsection[title={Optimal Estimation Retrievals with COSMO-7 Prior}]

    

\stopsubsection

\startsubsection[title={Optimal Estimation Retrievals with Climatology Prior}]

    First guess from regression

\stopsubsection

\stopsection


\startsection[title={Case Studies with Actual Radiometer Data}]

...

\startsubsection[title={Ground-based Temperature Inversion}]

    2015-10-28 02:15:05

\stopsubsection

\startsubsection[title={Elevated Temperature Inversion}]

    2015-09-11 03:48:00

\stopsubsection

\startsubsection[title={Mid-Troposphere Temperature Inversion}]

    2015-10-21 02:15:04

\stopsubsection

\startsubsection[title={Boundary Layer Evolution}]

    Start from elevated temperature inversion case.

\stopsubsection

\stopsection

